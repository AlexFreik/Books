\newpage
\section*{Руководство к проведению онлайн сессий}
\label{sec:online}
\addcontentsline{toc}{section}{\nameref{sec:online}}

После прохождения онлайн-тренинга Йога Вира, у вас есть возможность проводить следующие модули онлайн.
\begin{itemize}
\item Йога для иммунитета (Саштанга и Симха крийя).
\item Саштанга для благополучия (Йога Намаскар и Нади Шуддхи)
\item Медитация для начинающих (Иша крийя)
\end{itemize}
Во время сессии вы можете поделиться с участниками \textbf{ссылкой на видео}, используя окно чата платформы для видеоконференций. Воспользуйтесь опцией
«Поделиться экраном» внутри платформы, которую вы используете для
проведения сессии, и воспроизведите видео.
Для каждой сессии нужно будет воспроизводить только одно видео.
Пожалуйста, выберите вариант сессии ниже и используйте подходящее видео по соответствующей ссылке. Вы можете использовать информацию о длительности видео (см. ниже), чтобы определить подходящий модуль для вашей сессии.
После выбора модуля, пожалуйста, \textbf{следуйте руководству} ниже.

\subsection*{Руководство}
\label{sec:list}
\addcontentsline{toc}{subsection}{\nameref{sec:list}}

\begin{enumerate}
    \item \textbf{Ознакомьтесь} с тем, какие виды сессий вы можете проводить самостоятельно.
    \item \textbf{Обратитесь за поддержкой} в Телеграмм-группу Йога Виры, если она вам
необходима.
\item \textbf{Предоставьте ссылки на часто задаваемые вопросы}, если они возникнут:
\begin{itemize}
    \item Иша крийя: \href{https://bit.ly/3fEbs13}{\tiny https://bit.ly/3fEbs13}
    \item Симха крийя: \href{https://isha.sadhguru.org/global/ru/simha-kriya/faq}{\tiny https://isha.sadhguru.org/global/ru/simha-kriya/faq}
    \item Саштанга: \href{https://drive.google.com/file/d/1LDbTnDl8uo3g3QCbMTGpZU-Kw2URPeEH/view?usp=sharing}{\tiny https://drive.google.com/file/d/1LDbTnDl8uo3g3QCbMTGpZU-Kw2URPeEH/view?usp=sharing}
\end{itemize}
\item \textbf{Отправьте напоминание участникам о следующем}:
    \begin{itemize}
    \item Приступайте к сессии на пустой желудок для практик Саштанга и Симха крийя.
    \item Приступайте к сессии на полупустой желудок для практик Йога Намаскар и Нади Шуддхи.
    \item Условие полупустого желудка рекомендовано, но не обязательно для сессии Иша крийи.
    А именно:
    \begin{itemize}
        \item[\faClockO] Должно пройти минимум 1,5 часа после полноценного приема пищи для 40/60/90 мин. сессий Иша Упа-йоги.
        \item[\faClockO] Должно пройти минимум 2,5 часа после полноценного приема пищи для сессий Симха крийи.
        \item[\faClockO] Должно пройти минимум 4 часа после полноценного приема пищи для сессий Саштанги.
    \end{itemize}
    \end{itemize}

\item \emph{Принимайте участие во всей сессии целиком.} Попросите всех участников участвовать во всей сессии целиком и оставаться до самого конца. Включая друзей и членов семьи \faSmileO.
\item \emph{Убедитесь, что вас ничто не отвлекает.} Важно, чтобы в течение следующих <упомянуть длительность (например, 40 минут)> не было никаких
помех. Это включает в себя использование телефона и походы в туалет. Мы
хотели бы начать сессию со знакомства с Садхгуру, а затем посмотреть
видео, где он объясняет практику и знакомит нас с ней. Пожалуйста,
следуйте инструкциям на протяжении всего видео.
\item \textbf{Используйте \href{https://drive.google.com/file/d/19X10ANk28EUXqd9_cOFGla0Y-XeFVKd2/view?usp=sharing}{этот сценарий}} для проведения вашей сессии.

\end{enumerate}

\paragraph{После сессии}
\begin{enumerate}
    \setcounter{enumi}{8}
\item \textbf{Поделитесь ссылкой на онлайн-программу «Внутренняя инженерия»}
  \href{https://www.innerengineering.com/ru/online}{https://www.innerengineering.com/ru/online} \faSmileO. Вы можете выбрать такой стиль общения, который будет уместен для ваших друзей и близких.
    \item \textbf{Благодарность.} И, наконец, не забудьте поблагодарить их за участие и предоставленную вам возможность предложить им несколько минут йоги.
    \item \textbf{Заполните \href{https://forms.gle/q1N7jG4vBEWBmng86}{отчет}} после сессии. 
\end{enumerate}

\paragraph{Чего \emph{не} делать}
\begin{itemize}
    \item[\faRemove] \textbf{\emph{Не} создавайте собственную сессию или последовательность сессии.} Если у вас есть какие-то идеи, которые вы хотели бы рассмотреть, пожалуйста, сообщите нам sadhanasupport.russian@ishafoundation.org. \textbf{Пожалуйста, не применяйте} собственную последовательность и не предлагайте никаких других практик, которые не упомянуты в этом Приложении.
    \item[\faRemove] \textbf{\emph{Не} отвечайте на вопросы самостоятельно.} Если вы не прошли официальную подготовку для ответов на вопросы, пожалуйста, запишите все заданные вопросы, отправьте их по адресу sadhanasupport.russian@ishafoundation.org для того, чтобы получить ответы, и вернитесь с ответом к тем, кто задал вопросы.
    \item[\faRemove] \textbf{\emph{Не} обучайте ничему самостоятельно:} ни путем демонстраций практик, ни путем объяснений. Если у вас есть какие-то идеи, которые вы хотели бы
рассмотреть, пожалуйста, сообщите нам.
\end{itemize}

\begin{table}[ht!]
\centering
\begin{tabular}{||p{0.7\linewidth} | c c ||}
 \hline
 \rowcolor{lightgray} \textbf{Сессия} & \textbf{Видео} & \textbf{Время видео}  \\ [0.5ex]
 \hline\hline
\text{Йога для благополучия (40 мин.)}  \text{Йога Намаскар + Нади Шуддхи} & \href{https://drive.google.com/file/d/1NvL2jsmD-FzAOsnEUKNxM2n5dGn3bLbS/view}{ссылка} & 37 мин \\
\hline
\text{Йога для иммунитета (45 мин.)}  \text{Саштанга + Симха крийя} & \href{https://drive.google.com/file/d/1BO8APukLrjHJAYwDOMhp6qvGPZEh2jN7/view?usp=sharing}{ссылка} & 24 мин \\
\hline
\text{Медитация для начинающих (60 мин.)}  \text{Иша крийя\ \ \ \ \ \ } & \href{https://drive.google.com/file/d/1O3jdmsOFZ-kkBb_ZGFP7did8Yvhk-yMr/view?usp=sharing}{ссылка} & 36 мин \\
\hline
\text{Иша Упа йога (65 мин.)} & & \\ Движение рук по направлениям + Практики для шеи + Йога Намаскар + Нади Шуддхи & \href{https://drive.google.com/file/d/1O04lLBSqTakuITimFwo7eY6tpjlL3JV-/view?usp=sharing}{ссылка} & 63 мин \\
\hline
\text{Иша Упа йога (90 мин.)} & & \\
Движение рук по направлениям + Практики для шеи +
Йога Намаскар + Нади Шуддхи + Нада Йога + Шамбхави Мудра\footnote{Это не то же самое, что и Шамбхави Махамудра крийя.} + Практика Намаскар & \href{https://www.youtube.com/watch?v=Gseq7N49-JI}{ссылка} & 86 мин \\
\hline
\end{tabular}
\caption{Структура сессий}
\label{table:1}
\end{table}

\addcontentsline{toc}{subsection}{\nameref{table:1}}

\subsection*{Профессиональный совет: простая платформа}
\label{sec:profAdv1}
\addcontentsline{toc}{subsection}{\nameref{sec:profAdv1}}

Вы можете использовать ту платформу, которая вам удобна. Если вы никогда раньше не пользовались платформами для видеоконференций, Google Meet — отличное бесплатное программное обеспечение, которое вы можете использовать. 

\href{https://drive.google.com/file/d/1aqEODSOCmS3BdFkcbs-sf_Aa1cjpd37d/view?usp=sharing}{Видео инструкции по использованию GoogleMeet.}

Если у вас есть платный аккаунт в Zoom, вы также можете использовать эту платформу для проведения сессий.

\href{https://drive.google.com/file/d/1aqEODSOCmS3BdFkcbs-sf_Aa1cjpd37d/view?usp=sharing}{Видео инструкции по использованию Zoom.}

\subsection*{Профессиональный совет: продвижение в социальных сетях}
\label{sec:profAdv2}
\addcontentsline{toc}{subsection}{\nameref{sec:profAdv2}}

Социальные сети предоставляют различные способы продвижения сессий в Интернете. Это важный шаг в организации сессии, потому что, если вы не обратитесь к аудитории, никто не узнает, что вы проводите сессию!
Если вы новичок в социальных сетях, не волнуйтесь \faSmileO. Пожалуйста, ознакомьтесь с рекомендациями ниже:
\begin{enumerate}
    \item \textbf{WhatsApp/Telegram.} Поделитесь деталями вашей сессии вместе со ссылкой на присоединение к группе через рассылку WhatsApp/Telegram.
    \item \textbf{VK/Facebook/Instagram.} Расскажите о своем мероприятии через пост И поделитесь деталями мероприятия через Сторис в Инстаграм в течение нескольких дней, предшествующих мероприятию.
\end{enumerate}


\faLightbulbO\ Если вы готовы, то самый эффективный способ обратиться к аудитории — через VK/Instoagram Live.

\subsection*{Редактируемые приглашения: йога из дома}
\label{sec:invites}
\addcontentsline{toc}{subsection}{\nameref{sec:invites}}
Важно не редактировать первоначальные шаблоны. Пожалуйста, используйте \href{https://docs.google.com/file/d/1sdxy89oBTJ6PNSrEM1bYZ5grWjE6ot-6/edit?usp=docslist_api&filetype=msword}{это руководство} по использованию шаблонов. 
