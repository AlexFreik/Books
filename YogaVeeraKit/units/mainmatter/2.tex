\newpage
\section*{Руководство к проведению онлайн сессий}
\label{sec:online}
\addcontentsline{toc}{section}{\nameref{sec:online}}

После прохождения онлайн-тренинга Йога Вира, у вас есть возможность проводить следующие модули онлайн:  % edit
\begin{itemize}
\item <<Йога для иммунитета>> (Саштанга и Симха крийя);
\item <<Йога для благополучия>> (Йога Намаскар и Нади Шуддхи);
\item <<Медитация для начинающих>> (Иша крийя);
\item Иша Упа йога (60 и 90 мин).

    Мы рекомендуем первые три сессии, так как большинство зрителей предпочитают менее 1 часа. Но если у них есть заинтересованная аудитория, они могут проводить и более длительные выступления.
\end{itemize}


Во время сессии вы можете поделиться с участниками \textbf{ссылкой на видео}, используя окно чата платформы для видеоконференций. Воспользуйтесь опцией
«Поделиться экраном» внутри платформы, которую вы используете для
проведения сессии, и воспроизведите видео.
Для каждой сессии нужно будет воспроизводить только одно видео.

Пожалуйста, выберите вариант сессии ниже и используйте подходящее видео по соответствующей ссылке. Вы можете использовать информацию о длительности видео (см. ниже), чтобы определить подходящий модуль для вашей сессии.
После выбора модуля, пожалуйста, \textbf{следуйте руководству} ниже.

\subsection*{Руководство}
\label{sec:list}
\addcontentsline{toc}{subsection}{\nameref{sec:list}}

\begin{enumerate}
    \item \textbf{Ознакомьтесь} с тем, какие виды сессий вы можете проводить самостоятельно.
    \item \textbf{Обратитесь за поддержкой} в Телеграмм-группу Йога Виры, если она вам
необходима.
\item \textbf{Предоставьте ссылки на часто задаваемые вопросы}, если они возникнут:
\begin{itemize}
    \item Иша крийя: \href{https://bit.ly/3fEbs13}{\tiny https://bit.ly/3fEbs13}
    \item Симха крийя: \href{https://isha.sadhguru.org/global/ru/simha-kriya/faq}{\tiny https://isha.sadhguru.org/global/ru/simha-kriya/faq}
    \item Саштанга: \href{https://drive.google.com/file/d/1LDbTnDl8uo3g3QCbMTGpZU-Kw2URPeEH/view?usp=sharing}{\tiny https://drive.google.com/file/d/1LDbTnDl8uo3g3QCbMTGpZU-Kw2URPeEH/view?usp=sharing}
\end{itemize}
\item \textbf{Отправьте напоминание участникам о следующем}:
    \begin{itemize}
    \item Приступайте к сессии на пустой желудок для практик Саштанга и Симха крийя.
    \item Приступайте к сессии на полупустой желудок для практик Йога Намаскар и Нади Шуддхи.
    \item Условие полупустого желудка рекомендовано, но не обязательно для сессии Иша крийи.
    А именно:
% \newpage
% \pagestyle{empty}
    \begin{itemize}
        \item[\faClockO] Должно пройти минимум 1,5 часа после полноценного приема пищи для 40/60/90 мин. сессий Иша Упа-йоги.
        \item[\faClockO] Должно пройти минимум 2,5 часа после полноценного приема пищи для сессий Симха крийи.
        \item[\faClockO] Должно пройти минимум 4 часа после полноценного приема пищи для сессий Саштанги.
    \end{itemize}
    \end{itemize}

\item \emph{Принимайте участие во всей сессии целиком.} Попросите всех участников участвовать во всей сессии целиком и оставаться до самого конца. Включая друзей и членов семьи \faSmileO.
\item \emph{Убедитесь, что вас ничто не отвлекает.} Важно, чтобы в течение следующих <упомянуть длительность (например, 40 минут)> не было никаких
помех. Это включает в себя использование телефона и походы в туалет. Мы
хотели бы начать сессию со знакомства с Садхгуру, а затем посмотреть
видео, где он объясняет практику и знакомит нас с ней. Пожалуйста,
следуйте инструкциям на протяжении всего видео.
\item \textbf{Используйте \hyperref[sec:plan]{этот сценарий}} для проведения вашей сессии.
% https://drive.google.com/file/d/1LDbTnDl8uo3g3QCbMTGpZU-Kw2URPeEH/view
\end{enumerate}

\paragraph{После сессии}
\begin{enumerate}
    \setcounter{enumi}{8}
\item \textbf{Поделитесь ссылкой на онлайн-программу «Внутренняя инженерия»}
  \href{https://www.innerengineering.com/ru/online}{https://www.innerengineering.com/ru/online} \faSmileO. Вы можете выбрать такой стиль общения, который будет уместен для ваших друзей и близких.
    \item \textbf{Благодарность.} И, наконец, не забудьте поблагодарить их за участие и предоставленную вам возможность предложить им несколько минут йоги.
    \item \textbf{Заполните \href{https://forms.gle/q1N7jG4vBEWBmng86}{отчет}} после сессии. 
\end{enumerate}

\paragraph{Чего \emph{не} делать}
\begin{itemize}
    \item[\faRemove] \textbf{\emph{Не} создавайте собственную сессию или последовательность сессии.} Если у вас есть какие-то идеи, которые вы хотели бы рассмотреть, пожалуйста, сообщите нам sadhanasupport.russian@ishafoundation.org. \textbf{Пожалуйста, не применяйте} собственную последовательность и не предлагайте никаких других практик, которые не упомянуты в этом Приложении.
    \item[\faRemove] \textbf{\emph{Не} отвечайте на вопросы самостоятельно.} Если вы не прошли официальную подготовку для ответов на вопросы, пожалуйста, запишите все заданные вопросы, отправьте их по адресу sadhanasupport.russian@ishafoundation.org для того, чтобы получить ответы, и вернитесь с ответом к тем, кто задал вопросы.
    \item[\faRemove] \textbf{\emph{Не} обучайте ничему самостоятельно:} ни путем демонстраций практик, ни путем объяснений. Если у вас есть какие-то идеи, которые вы хотели бы рассмотреть~--- пожалуйста, сообщите нам.
\end{itemize}

\subsection*{Структура}
\label{sec:struct}
\addcontentsline{toc}{subsection}{\nameref{sec:struct}}

Для сессий используйте видео, доступные по ссылкам в перечне ниже. Также все видео-инструкции можно найти в \href{https://youtube.com/playlist?list=PLnqgRgprlYQgJwgTSo8M29l0aJHQG2Efg}{этом плейлисте} на официальном YouTube канале Садхгуру.
\begin{itemize}
\item \href{https://drive.google.com/file/d/1OKuzlk67PiygUcywCozbgIGB38TLH6sv/view?usp=sharing}{Йога для иммунитета (видео 24 мин)}: 

    Саштанга + Симха крийя

\item \href{https://drive.google.com/file/d/1ONdlaZQIkNHkhtjV2z1Zkuut3AXz9Fys/view?usp=sharing}{Медитация для начинающих (видео 36 мин)}: 

    Иша крийя

\item \href{https://drive.google.com/file/d/1ONYEw1Z5vYU1K8JGLy2UkQTufU0hSr7u/view?usp=sharing}{Йога для благополучия (видео 37 мин)}: 

    Йога Намаскар + Нади Шуддхи


\item \href{https://drive.google.com/file/d/1O04lLBSqTakuITimFwo7eY6tpjlL3JV-/view?usp=sharing}{Иша Упа йога (видео 63 мин)}: 

    Движение рук по направлениям + Практики для шеи + Йога Намаскар + Нади Шуддхи
    

\item \href{https://youtu.be/Gseq7N49-JI}{Иша Упа йога (видео 86 мин)}: 

    Движение рук по направлениям + Практики для шеи + Йога Намаскар + Нади Шуддхи + Нада Йога + Шамбхави Мудра\footnote{Это не то же самое, что и Шамбхави Махамудра крийя.} + Практика Намаскар
\end{itemize}

% \begin{table}[ht!]
% \centering
% \begin{tabular}{|| p{0.48\linewidth} p{0.42\linewidth} | p{0.09\linewidth} ||}
%  \hline
%  \rowcolor{lightgray} \textbf{Сессия} & \textbf{Практики} & \textbf{Длина видео}  \\ [0.5ex]
% \hline\hline
% \href{https://drive.google.com/file/d/1OKuzlk67PiygUcywCozbgIGB38TLH6sv/view?usp=sharing}{Йога для иммунитета (45 мин)} & \href{https://youtu.be/RkWUAzPPuZ0}{Саштанга + Симха крийя} & 24 мин \\
% \hline
% \href{https://drive.google.com/file/d/1ONdlaZQIkNHkhtjV2z1Zkuut3AXz9Fys/view?usp=sharing}{Медитация для начинающих (60 мин)} &  \href{https://youtu.be/1VKDQraF82Y}{Иша крийя} & 36 мин \\
% \hline
% \href{https://drive.google.com/file/d/1ONYEw1Z5vYU1K8JGLy2UkQTufU0hSr7u/view?usp=sharing}{Йога для благополучия (40 мин)} & \href{https://youtu.be/aRxBafYDFo8}{Йога Намаскар} + \href{https://youtu.be/nfukQQqCM44}{Нади Шуддхи} & 37 мин \\
% \hline
% \href{https://drive.google.com/file/d/1O04lLBSqTakuITimFwo7eY6tpjlL3JV-/view?usp=sharing}{Иша Упа йога (65 мин)} &  \href{https://youtu.be/y-7UzbyJrR4}{Движение рук по направлениям} + \href{https://youtu.be/yuDuHILVl5Q}{Практики для шеи} + \href{https://youtu.be/aRxBafYDFo8}{Йога Намаскар} + \href{https://youtu.be/nfukQQqCM44}{Нади Шуддхи} & 63 мин \\
% \hline
% \href{https://youtu.be/Gseq7N49-JI}{Иша Упа йога (90 мин)} & \href{https://youtu.be/y-7UzbyJrR4}{Движение рук по направлениям} + \href{https://youtu.be/yuDuHILVl5Q}{Практики для шеи} + \href{https://youtu.be/aRxBafYDFo8}{Йога Намаскар} + \href{https://youtu.be/nfukQQqCM44}{Нади Шуддхи} + \href{https://youtu.be/vgA0W_FR5Ps}{Нада Йога} + \href{https://youtu.be/yvqJQDsw4bE}{Шамбхави Мудра\footnote{Это не то же самое, что и Шамбхави Махамудра крийя.}} + \href{https://youtu.be/Yv1J3QHObu8}{Практика Намаскар} 
% & 86 мин \\
% \hline
% \end{tabular}
% \caption{Структура сессий}
% \label{table:1}
% \end{table}


\subsection*{Профессиональный совет: простая платформа}
\label{sec:profAdv1}
\addcontentsline{toc}{subsection}{\nameref{sec:profAdv1}}

Вы можете использовать ту платформу, которая вам удобна. Если вы никогда раньше не пользовались платформами для видеоконференций, Google Meet — отличное бесплатное программное обеспечение, которое вы можете использовать. 

\href{https://drive.google.com/file/d/1aqEODSOCmS3BdFkcbs-sf_Aa1cjpd37d/view?usp=sharing}{Видео инструкции по использованию GoogleMeet.}

Если у вас есть платный аккаунт в Zoom, вы также можете использовать эту платформу для проведения сессий.

\href{https://drive.google.com/file/d/10llvQ_0aU7aWvye_Qvgj59LNMaya3aF8/view?usp=sharing}{Видео инструкции по использованию Zoom.}

\subsection*{Профессиональный совет: продвижение в социальных сетях}
\label{sec:profAdv2}
\addcontentsline{toc}{subsection}{\nameref{sec:profAdv2}}

Социальные сети предоставляют различные способы продвижения сессий в Интернете. Это важный шаг в организации сессии, потому что, если вы не обратитесь к аудитории, никто не узнает, что вы проводите сессию!
Если вы новичок в социальных сетях, не волнуйтесь \faSmileO. Пожалуйста, ознакомьтесь с рекомендациями ниже:
\begin{enumerate}
    \item \textbf{WhatsApp/Telegram.} Поделитесь деталями вашей сессии вместе со ссылкой на присоединение к группе через рассылку WhatsApp/Telegram.
    \item \textbf{VK/Facebook/Instagram.} Расскажите о своем мероприятии через пост И поделитесь деталями мероприятия через Сторис в Инстаграм в течение нескольких дней, предшествующих мероприятию.
\end{enumerate}


\faLightbulbO\ Если вы готовы, то самый эффективный способ обратиться к аудитории — через VK/Instagram Live.

\subsection*{Редактируемые приглашения: йога из дома}
\label{sec:invites}
\addcontentsline{toc}{subsection}{\nameref{sec:invites}}
Важно не редактировать первоначальные шаблоны. Пожалуйста, используйте \hyperref[sec:templates]{это руководство} по использованию шаблонов. 

Ниже приведены ссылки на шаблоны. Просто вставьте их в браузер и следуйте инструкции из \hyperref[sec:templates]{руководства}.
\begin{enumerate}\label{sec:templatesRefs}
    \item Шаблон для "Йоги для благополучия"\ \href{https://ishaeu.org/YVTemplateRU1}{\small https://ishaeu.org/YVTemplateRU1} 
    \item Шаблон для "Медитации для начинающих"\ \href{https://ishaeu.org/YVTemplateRU2}{\small https://ishaeu.org/YVTemplateRU2}
    \item Шаблон для "Йоги для иммунитета"\ \href{https://ishaeu.org/YVTemplateRU3}{\small https://ishaeu.org/YVTemplateRU3}  

\end{enumerate}

