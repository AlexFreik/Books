\newpage
\section*{Чек-лист для проведения сессии}
\label{sec:check}
\addcontentsline{toc}{section}{\nameref{sec:check}}

\subsection*{После программы подготовки}
\begin{itemize}
\item Проверьте свое интернет-соединение и его скорость. Убедитесь, что соединение стабильно и надежно. (Воспользуйтесь \href{https://www.speedtest.net}{данной ссылкой}, чтобы протестировать его.)

\item Проведите пробную сессию всего обучающего занятия целиком с другим Йога Вирой/волонтером. Используйте для неё ту же платформу, которую вы планируете использовать для запланированной обучающей сессии. Протестируйте свой микрофон и камеру, а также воспроизведите все видео, чтобы проверить трансляцию.

\end{itemize}

\subsection*{Перед каждой сессией}
\begin{itemize}
\item Убедитесь, что ноутбук/настольный компьютер подключен к питанию и есть запасной блок питания.
\item Проверьте аудио и видео оборудование.
\item Убедитесь, что у вас есть все необходимые файлы для сессии.
\item Убедитесь, что вы расслаблены и готовы к проведению сессии \faSmileO.
\end{itemize}
\subsection*{Во время сессии}
\begin{itemize}
\item Убедитесь, что микрофоны всех участников переведены в бесшумный режим, чтобы избежать фонового шума со стороны других участников.

\item Вы можете попросить участников закрепить окно с вашим изображением для обеспечения непрерывной трансляции видео.

\item Сделайте скриншот экрана во время одного из видео, чтобы было видно количество участников сессии.

\item Отправьте участникам \href{https://forms.gle/oxQNwrThpZ5bxwFm8}{форму «Оставайтесь на связи»} в конце сессии. 
\end{itemize}

\subsection*{После сессии}
\begin{itemize}
\item Заполните \href{https://forms.gle/q1N7jG4vBEWBmng86}{«Отчёт Йога Виры»} с количеством участников сессии, скриншотом и т.д.

\item Запланируйте следующую онлайн-сессию \faSmileO. 
\end{itemize}
