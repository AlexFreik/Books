\newpage
\section*{Сценарий сессии}
\label{sec:plan}
\addcontentsline{toc}{section}{\nameref{sec:plan}}

\textbf{15 минут до начала сессии~---} откройте трансляцию (в зависимости от используемой платформы). 

\textbf{Ровно в назначенное время начала сессии~---} поприветствуйте участников, подождете пока все проверят свой звук и видео, переведите микрофоны (и видео) участников в бесшумный режим.

\textbf{Объявление, пока присоединяются другие участники}
\begin{quote}\emph{
Намаскарам/Доброе утро/Добрый день/Добрый вечер!
Добро пожаловать на сегодняшнюю сессию. Мы подождём ещё несколько
минут пока остальные участники присоединяются. Пока мы ждём, будет
лучше, если вы будете использовать наушники для участия в сессии и
оставите свои микрофоны в бесшумном режиме. (Если участникам нужна
помощь, вы можете сами перевести их микрофоны в бесшумный режим,
либо показать им как это сделать).
\\[3pt]
Доброе утро/день/вечер и добро пожаловать на сегодняшнюю
йога-сессию.
}\end{quote}

\textbf{На выбор} (вы можете представиться, если участники с вами не знакомы)
\begin{quote}\emph{
Меня зовут \underline{\qquad} и я волонтёр Фонда «Иши» — международной
некоммерческой организации, основанной Садхгуру — йогином и
мистиком. Садхгуру предлагает мощные методы для само-преображения и
расширения своих возможностей через науку о йоге.
Я (назвать свою профессию: программист, инженер, домохозяйка, студент
итд) и на данный момент я работаю в \underline{\qquad}.
}\end{quote}

\begin{quote}\emph{
Сегодня я здесь для того, чтобы предложить вам простой набор практик,
спроектированных Садхгуру, которые позволят достичь физического и
психологического благополучия.
\\[3pt]
Продолжительность этой сессии \underline{\qquad} минут.
\\[3pt]
Чтобы извлечь максимальную пользу из своего участия в этой сессии, есть несколько вещей, которые вы можете сделать, чтобы создать благоприятную атмосферу. Пожалуйста, обратите внимание, что очень важно пройти эту сессию от начала до конца, так как предлагаемые практики должны изучаться систематически. Также мы просим вас не прерываться во время сессии на походы в туалет или использование телефона и не вставать.
\\[3pt]
\textbf{(В зависимости от того, какая сессия проводится).} Определенные практики требуют, чтобы их выполняли на полупустой желудок — должно пройти 1.5 часа после полноценного приема пищи (для Йога Намаскар) и 2.5 часа (для Симха крийи). Если сейчас соблюдение этого условия для вас невозможно, пожалуйста, не выполняйте эти практики и сделайте их, когда условие полупустого желудка будет соблюдено.
\\[3pt]
Сейчас Садхгуру даст необходимые рекомендации для практик. Сперва мы покажем короткое введение, а затем приступим к самим практикам. Сначала мы по смотрим демонстрацию, а затем перейдём к инструкциям для практик.
\\[15pt]
\textbf{Воспроизведите видео с практиками}
\\[15pt]
Мы надеемся, что вам понравилась сегодняшняя сессия. Эти практики
доступны на официальных каналах Садхгуру в YouTube 
\\ \href{https://youtube.com/playlist?list=PLnqgRgprlYQgJwgTSo8M29l0aJHQG2Efg}{\tiny https://youtube.com/playlist?list=PLnqgRgprlYQgJwgTSo8M29l0aJHQG2Efg} . Просто следуйте
инструкциям из видео до тех пор, пока вам не станет комфортно выполнять
практики самостоятельно.
\\[3pt]
Если у вас есть какие-либо вопросы по практикам, вы можете написать на
электронную почту: sadhanasupport.russian@ishafoundation.org \textbf{(скиньте адрес электронной почты в чат трансляции)}.
\\[3pt]
Если вы хотите исследовать науку о йоге во всей ее полноте и измерениях,
то следующий шаг, который мы рекомендуем предпринять, это наша
ведущая программа «Внутренняя инженерия». Онлайн-курс содержит 7
онлайн сессий, которые вы можете пройти вместе с Садхгуру в своём
темпе прямо у себя дома. Каждая сессия продолжительностью примерно
1,5 часа. После этого вы можете принять участие в Завершающей
программе, 3-дневном онлайн курсе, который включает передачу
Шамбхави Махамудра крийи — мощной очищающей 21-минутной
энергетической практики, которая объединяет дыхание вместе с
оздоравливающими и укрепляющими асанами.
\\[3pt]
Если вы хотите узнать подробности программы, посетите официальный
сайт: \href{http://www.innerengineering.com}{\small http://www.innerengineering.com} \textbf{(поделитесь ссылкой в чате трансляции)}.
}\end{quote}

\textbf{По желанию}
\begin{quote}\emph{
Я бы хотел(а) поделиться с вами своим опытом от прохождения программы «Внутренняя инженерия»
\textbf{(поделитесь своими впечатлениями и опытом от своей программы
«Внутренняя инженерия» согласно инструкциям, которые были даны во
время программы подготовки)}.
}\end{quote}

\textbf{По желанию}
\begin{quote}\emph{
Если вы хотите узнать больше о программе, пожалуйста, перейдите на сайт \href{http://www.innerengineering.com}{\small http://www.innerengineering.com} \textbf{(скиньте ссылку в чат трансляции)}. 
}\end{quote}

\begin{quote}\emph{
Спасибо за то, что приняли участие в сегодняшней сессии! Было очень приятно всех вас видеть. Если вы хотите узнать больше о подобных сессиях, вы можете найти их по ссылке, опубликованной в чате трансляции: \href{https://ishaeu.org/Yoga-veera-sessions-RU}{\small https://ishaeu.org/Yoga-veera-sessions-RU} \textbf{(скиньте ссылку в чат трансляции)}. 
}\end{quote}

