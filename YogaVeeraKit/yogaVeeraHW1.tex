\documentclass[
a4paper, % Stock and paper size.
12pt, % Type size.
article,
% oneside, 
onecolumn, % Only one column of text on a page.
% openright, % Each chapter will start on a recto page.
% openleft, % Each chapter will start on a verso page.
openany, % A chapter may start on either a recto or verso page.
]{memoir}

%%% PACKAGES 
%%%------------------------------------------------------------------------------

\usepackage[utf8]{inputenc} % If utf8 encoding
% \usepackage[lantin1]{inputenc} % If not utf8 encoding, then this is probably the way to go
\usepackage[T1]{fontenc}    %
\usepackage[english,russian]{babel} % English please
\usepackage[final]{microtype} % Less badboxes

% \usepackage{kpfonts} %Font

\usepackage{amsmath,amssymb,mathtools} % Math

%%% Figures and colors
\usepackage[usenames,dvipsnames,svgnames,table,rgb]{xcolor}
\usepackage{tikz} % Figures
\usepackage{graphicx}  % Include figures
\usepackage{wrapfig}
\graphicspath{{img/}{../img/}}

%%% PAGE LAYOUT 
%%%------------------------------------------------------------------------------

\setlrmarginsandblock{0.08\paperwidth}{*}{1} % Left and right margin
\setulmarginsandblock{0.15\paperwidth}{*}{1}  % Upper and lower margin
\checkandfixthelayout
%%% indenting
\setlength{\parindent}{0em}
\setlength{\parskip}{0.5em}

%%% INTERNAL HYPERLINKS
%%%-------------------------------------------------------------------------------

\usepackage{hyperref}   % Internal hyperlinks
\hypersetup{
pdfborder={0 0 0},      % No borders around internal hyperlinks
pdfauthor={I am the Author} % author
}
% ----------- hyper ref -----------
\usepackage{hyperref}


\newcommand{\linkcolor}{blue}
\newcommand{\citecolor}{blue}
\newcommand{\filecolor}{magenta}
\newcommand{\urlcolor}{NavyBlue}
\hypersetup{				% Гиперссылки
	unicode=true,           % русские буквы в раздела PDF\\
	pdfstartview=FitH,
	colorlinks=true,  % false: ссылки в рамках; true: цветные ссылки
	linkcolor=\linkcolor,         % внутренние ссылки
	citecolor=\citecolor,        % на библиографию
	filecolor=\filecolor,      % на файлы
	urlcolor=\urlcolor,      % на URL
}
\usepackage{memhfixc}   %



% Defining a new coordinate system for the page:
%
% --------------------------
% |(-1,1)    (0,1)    (1,1)|
% |                        |
% |(-1,0)    (0,0)    (1,0)|
% |                        |
% |(-1,-1)   (0,-1)  (1,-1)|
% --------------------------
\makeatletter
\def\parsecomma#1,#2\endparsecomma{\def\page@x{#1}\def\page@y{#2}}
\tikzdeclarecoordinatesystem{page}{
    \parsecomma#1\endparsecomma
    \pgfpointanchor{current page}{north east}
    % Save the upper right corner
    \pgf@xc=\pgf@x%
    \pgf@yc=\pgf@y%
    % save the lower left corner
    \pgfpointanchor{current page}{south west}
    \pgf@xb=\pgf@x%
    \pgf@yb=\pgf@y%
    % Transform to the correct placement
    \pgfmathparse{(\pgf@xc-\pgf@xb)/2.*\page@x+(\pgf@xc+\pgf@xb)/2.}
    \expandafter\pgf@x\expandafter=\pgfmathresult pt
    \pgfmathparse{(\pgf@yc-\pgf@yb)/2.*\page@y+(\pgf@yc+\pgf@yb)/2.}
    \expandafter\pgf@y\expandafter=\pgfmathresult pt
}
\makeatother

%%% for beautiful icons
\usepackage{fontawesome}


\begin{document}

\tikz[remember picture,overlay] \node[opacity=0.9,inner sep=0pt] at (page cs:0.8,0.8){\includegraphics[width=0.1\paperwidth]{IshaLogo}};

\section*{Инструкции к первой домашней работе}
\label{sec:homework}
\addcontentsline{toc}{section}{\nameref{sec:homework}}

\subsection*{Партнерская система}
\begin{enumerate}
\item группа в телеграмме будет создана примерно для 10-15 волонтеров йога вира.
\item ваш йога митра (координатор) назначит вам партнера, с которым вы будете работать в
паре.
\item в качестве домашнего задания, вы проведёте практическую сессию вместе со своим
партнером.
\item после того, как вы оба завершите проведение своей сессии, уведомите об этом йога митру.
\end{enumerate}

\subsection*{Структура}

\begin{enumerate}
\item Продолжительность сессии 20 минут.
\item Приготовьте 3 видео, которые вы будете использовать, заранее. пожалуйста, ознакомьтесь с процессом трансляции видео перед сессией. Воспользуйтесь \ref{sec:profAdv1}{обучающими видео} из документа  <<Набор Йога Виры>>.
\item Представьтесь в начале сессии, так как это описано в сценарии, который у вас есть.
(пожалуйста, максимально придерживайтесь текста сценария)
\item Воспроизведите вводное видео Cадхгуру. (вы можете остановить видео после 30
секунды.)
\item Воспроизведите видео практики. (вы можете остановить его после 30 секунды.)
\item Расскажите о программе «внутренняя инженерия» так, как это описано в сценарии.
\item Воспроизведите видео о программе <<внутренняя инженерия>>.
\item Завершите сессию так, как это описано в сценарии.
\end{enumerate}


\subsection*{Вы ведете практическую сессию}
\begin{itemize}
\item Проводите репетицию практической сессии так, как если бы вы проводили настоящую.
сессию. Не останавливайте и не прерывайте сессию, даже если вы совершили ошибку —
попробуйте найти решение как можно продолжить дальше \faSmileO.
\item Как только практическая сессия закончена, пожалуйста, уделите минуту, чтобы ответить на следующие вопросы и поделитесь ответами со своим партнером.
\begin{enumerate}
\item Какие у вас впечатления после проведения практической сессии? Что происходило
внутри вас?
\item Какие, по-вашему мнению, части сессии могли бы быть сделаны лучше?
\end{enumerate} 
\end{itemize}

\subsection*{Ваш партнер проводит сессию}
\begin{itemize}
\item Принимайте участие словно вы пришли на настоящую сессию в качестве участника. Не
перебивайте его в процессе занятия.
\item Создайте атмосферу доверия, чтобы ваш партнёр смог поделиться своими
впечатлениями и ощущениями без страха осуждения \faSmileO.
\item Если вы заметили что-то, что можно улучшить, убедитесь, что вы сами привнесете это в
свою сессию, когда будете ее проводить.
\item Поблагодарите и поддержите своего партнера после того, как он поделиться своими
ощущениями от проведённой сессии.
\end{itemize} 

\end{document}

