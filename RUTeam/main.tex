%&-job-name=newfilenameialwayswanted
\documentclass[
a4paper, % Stock and paper size.
12pt, % Type size.
article,
% oneside, 
onecolumn, % Only one column of text on a page.
% openright, % Each chapter will start on a recto page.
% openleft, % Each chapter will start on a verso page.
openany, % A chapter may start on either a recto or verso page.
]{memoir}
\synctex=1
\maxtocdepth{subsection}
\setsecnumdepth{subsection}
\counterwithout{section}{chapter}
%%% PACKAGES 
%%%------------------------------------------------------------------------------
\input{../preamble/rusBabel}
%%% Figures and colors
\usepackage[usenames,dvipsnames,svgnames,table,rgb]{xcolor}
\usepackage{tikz} % Figures
\usepackage{graphicx}  % Include figures
\usepackage{wrapfig}
\graphicspath{{img/}{../img/}} 


%%% INTERNAL HYPERLINKS
%%%-------------------------------------------------------------------------------

\usepackage{hyperref}   % Internal hyperlinks
\newcommand{\linkcolor}{blue}
\newcommand{\citecolor}{blue}
\newcommand{\filecolor}{magenta}
\newcommand{\urlcolor}{NavyBlue}
\hypersetup{				% Гиперссылки
    pdfborder={0 0 0},      % No borders around internal hyperlinks
	unicode=true,           % русские буквы в раздела PDF\\
	pdfstartview=FitH,
	colorlinks=true,  % false: ссылки в рамках; true: цветные ссылки
	linkcolor=\linkcolor,         % внутренние ссылки
	citecolor=\citecolor,        % на библиографию
	filecolor=\filecolor,      % на файлы
	urlcolor=\urlcolor,      % на URL
    linkbordercolor=\linkcolor,  % hyperlink border will be red 
    pdftitle={Yoga Veera Kit},
    pdfpagemode=FullScreen,
    pdfauthor={I am the Author} % author
}

\let\oldhref\href
\renewcommand{\href}[2]{\oldhref{#1}{\underline{#2}}}

   
\graphicspath{{img/}{../img/}{../FreqImg/}}


%%% PAGE LAYOUT 
%%%------------------------------------------------------------------------------

\setlrmarginsandblock{0.15\paperwidth}{*}{1} % Left and right margin
\setulmarginsandblock{0.10\paperwidth}{*}{1}  % Upper and lower margin
\checkandfixthelayout
%%% indenting
\setlength{\parindent}{0em}
\setlength{\parskip}{0.5em}
\begin{document}
\begin{center}
    \Huge \textbf{RU Team Guide}
\end{center}
\tableofcontents

\tikz[remember picture,overlay] \node[opacity=0.9,inner sep=0pt] at (page cs:0.8,0.8){\includegraphics[width=0.1\paperwidth]{IshaLogo}};

\section{YouTube Videos Translation Process}

\subsection{\ldots}
\ldots
\subsection{Создание аудио трека}
Подробнее должен рассказать Эльдар.
\subsection{Создание финального видео}
\subsubsection{AirTable}
Эльдар после создания аудио трека отмечает ответственного за этот этап в AirTable. В последнем можно будет найти ссылки на оригинальное видео (раздел <<URL ORIGINAL>>), дедлайн (<<GIVE TO EMEDIA BY>>) и аудио трек (<<MEDIA LINK>>). 

Так же там будут два поля с названиями: <<TN TITLE>> и  <<TITLE>>. {\color{gray}Разница в том, что первое используется для tumbneils (это картинка, которая будет показываться у этого видео в ленте, TN как раз сокращение от tumbneil). В то же время, <<TITLE>> используется для названия видеоролика. }

Монтажеру рекомендуется использовать <<TN TITLE>> при руссификации заставки, если это необходимо.

После того как видео смонтированно, \hyperref[montageRules]{подробнее об монтаже тут}, нужно загрузить получившийся продукт на гугл диск по ссылке <<ARCHIVAL FOLDER>>, после чего 
\begin{itemize}
    \item отметить ответственного за следующий этап (Александра Гринева) в поле <<COLLABORATORS>>, 
\item в <<VIDEO STATUS>> отметить <<Video ready>>.
\end{itemize}

В случае, если видео исправляется и перезаливается необходимо сообщить об этом ответственному за следующий этап (Александру Гриневу), чтобы новая версия так же была проверенна.

\subsection{\ldots}
\ldots


\newpage
\section{Фильтрация спама от AitTable}
\subsection{Gmail}
\begin{enumerate}
    \item Откройте любое \emph{нежелательное} письмо.
	    \begin{center} 
	        \includegraphics[width=0.9\textwidth]{AirTableSpam/gmail0} 
	    \end{center}
    \item Нажмите значек \textbf{"More"} и выберите 
                \textbf{"Filter messages like these"}.
	    \begin{center} 
	        \includegraphics[width=0.5\textwidth]{AirTableSpam/gmail1} 
	    \end{center}
    \item В открывшемся окне настройте параметры, как показано ниже. 
        Далее нажмите \textbf{"Create filter"}.
        \begin{itemize}
            \item From: \textbf{"noreply@airtable.com"}
            \item Subject: \textbf{"New comment on "}
        \end{itemize}

	    \begin{center} 
	        \includegraphics[width=0.9\textwidth]{AirTableSpam/gmail2} 
	    \end{center}
    \item Отметьте параметры \textbf{"Skip the Inbox (Archive it)"} и
        \textbf{"Mark as read"}, как показано ниже. 
        \begin{itemize}
            \item Далее все письма от \textbf{"noreply@airtable.com"}, 
                предмет которых которых содержит "New comment on "
                будут автоматически помечаться прочитанными и 
                перемещаться в папку \textbf{"Archive"}.
            \item Если отметить калочкой \textbf{"Also apply filter to ..."},
                то фильтр применится и к \emph{нежелательным} письмам, 
                которые уже пришли.
        \end{itemize}

	    \begin{center} 
	        \includegraphics[width=0.7\textwidth]{AirTableSpam/gmail3} 
	    \end{center}
\end{enumerate}


\newpage
\subsection{Yandex mail}
\begin{enumerate}
    \item Откройте любое \emph{нежелательное} письмо.
	    \begin{center} 
	        \includegraphics[width=0.9\textwidth]{AirTableSpam/ya0} 
	    \end{center}
    \item Нажмите значек \textbf{"More"} и выберите 
                \textbf{"Create filter"}.
	    \begin{center} 
	        \includegraphics[width=0.5\textwidth]{AirTableSpam/ya1} 
	    \end{center}
    \item В открывшемся окне настройте параметры, как показано ниже. 
        Далее нажмите \textbf{"Create filter"}.
        \begin{itemize}
            \item Не забудьте поменять \textbf{matches} на 
                \textbf{contains} в условии 
                фильтра по предмету письма и заменить текст на 
                \textbf{"New comment on "}.
            \item Если отметить калочкой \textbf{"Apply to existing messages"},
                то фильтр применится и к нежелательным письмам, которые 
                уже пришли.
            \item Далее все письма от \textbf{"noreply@airtable.com"}, предмет
                которых которых содержит \textbf{"New comment on "} 
                будут автоматически помечаться прочитанными 
                и перемещаться в папку \textbf{"Archive"}.
        \end{itemize}

	    \begin{center} 
	        \includegraphics[width=0.9\textwidth]{AirTableSpam/ya2} 
	    \end{center}

\end{enumerate}

\newpage
\section{Монтаж --- советы \& правила}\label{montageRules}
\subsection{Правила}
В правилах мы выделили наилучшие практики, которые позволяют выдерживать качество и созранять общий стиль.
\begin{enumerate}
\item Параметры экспорта:
    \begin{itemize}
    \item кодек --- H264;
    \item битрейт видео~--- 8mbps;
    \item битрейт аудио --- 320;
    \item разрешение --- $1920 \times 1080$;
    \end{itemize}

    Если разрешение оригинала $1280 \times 720$,
        то фиальное видео делаем в формате
    $1920 \times 1080$ {\color{gray}(рекомендательные алгоритмы 
        YouTube поощряют 
    высокоформатные видео)}. 

    Но, если качесво $480$\ или ниже --- то оставляем 
    качесво оригинала {\color{gray}(растягивать $480$ до $1080$ уже перебор)}.


\item У широких видео дублируем его-же на бекграунд, растягивем до границ, и накладываем блюр.

	\begin{center} \includegraphics[width=0.5\textwidth]{tooWide} \end{center}

\item Устаревшие заставки <<Sadhguru. Yogi, mystic and visioner.>> и <<Conversation with Mystic>> вырезаются. {\color{gray}Они только отнимает время.}

\item Конечный слайд нужно заменять на \href{https://drive.google.com/file/d/11NbSgvq8LbxDcy-a2WY5OJTKUZKcZx88/view?usp=sharing}{русский}. 

	Конечная надпись <<\textcopyright\ Sadhguru 2021>> не меняется.

 Часто используемые слайды на 
\href{https://drive.google.com/drive/folders/1O54z3DtKpl90ut0Aa8wYkEEP37e00zPY?usp=sharing}{Google Drive}.

\item Вставки с текстом (например, вопросов). Если текст не озвучивается, то время фрагмента с текстом должно быть достаточным, чтобы вы могли неспеша прочитать его (возможно, придется удлиннить видео).

\item Английский текст при переводе не должен быть виден. В случае сложной анимации, этим можно пренебречь в угоду общей эстетики.

\item Любой сколько-нибудь сложный перевод, особенно если есть какие-то названия, нужно согласовать с отвественным человеком (Юрий Кузмин или Ксения Ошерова).

	Перевод вопросов и т.д. обычно имеется у звукоря.

\item Дизайнерский отдел Иши дал следующие шрифты: Merriweather для заголовков, крупного текста и Open Sans для субтитров. Используем их.

    Иногда используется Segoe Script, вы его сразу заметите.

 \includegraphics[width=0.3\textwidth]{segoeScript}

    Их можно скачать и установить из Google Fonts:
    \begin{itemize}
        \item  Merriweather \href{https://fonts.google.com/specimen/Merriweather}{\small https://fonts.google.com/specimen/Merriweather};
        \item Open Sans \href{https://fonts.google.com/specimen/Open+Sans}{\small https://fonts.google.com/specimen/Open+Sans};
    \item Segoe Script \href{https://www.fonts.com/font/microsoft-corporation/segoe-script?QueryFontType=Web&src=GoogleWebFonts}{\small https://www.fonts.com/font/microsoft-corporation/segoe-script?QueryFontType=Web\&src=GoogleWebFonts}.
      \end{itemize}


\item Переходы между фрагментами с разным фоном через Dip to Black, иначе через Dissolve.

\item Перевод названия в начале. Если картинка без особых движений обычно фиксирую первый кадр (когда англ текст ещё не появился) и добавить свой. 
	Если есть красивая анимация, то блюрю фона предпочтительнее непрозрачного прямоугольника.

\begin{figure}[!hbp]
  \centering
  \begin{minipage}[b]{0.4\textwidth}
    \includegraphics[width=\textwidth]{titleBlur}
    \caption{Nice.}
  \end{minipage}
  \hfill
  \begin{minipage}[b]{0.4\textwidth}
    \includegraphics[width=\textwidth]{titleBlurBad}
    \caption{Not nice.}
  \end{minipage}
\end{figure}

\item Перевод / добавление субтитров. По ситуации~--- обычно блюр красивее непрозрачного фона, но иногда нет. 

\item Текст без особого смысла, вроде "Darshan — Dec 2012
   Isha Yoga Center" оставляем как есть, наше время ценнее =).
\end{enumerate}

\subsection{Советы}
\begin{enumerate}

\item Для пользователей MacOS самый простой вариант монтирующей системы~--- iMovie, но у нее есть три серьезных недостатка.
    \begin{itemize}
    \item Нет возможности добавить текст поверх видео. То есть, придется делать картинку с перевдодм в отдельном приложении и наклеивать поверх. Это нарушает поток, и геморно исправлять текст.
    \item Можно использовать только два видео трека. Это ограничивает более сложных монтаж, хотя обычно двух хватает.
    \item Нет возможности почеловечески кастомизировать разрешение и формат видео. 
\end{itemize}
Поэтому совет~--- скачать более профессиональное приложение. В ашраме в основном используется Adobe Premiere Pro, насколько я слышал. 

Чтобы его спиратить, наберите в Youtube "Premiere Pro download MacOS free" и выберите видео с большим числом просмотров и положительными отзывами.
\end{enumerate}



\subsection{Test Task (Тренировочное видео}
\href{https://www.youtube.com/watch?v=9sGJUR7stzc}{Оригинал.}
\quad
\href{https://drive.google.com/file/d/1Y6ECjMSvkaUFmNawIePfFvqS2ZnB3SPi/view?usp=sharing}{Аудио трек.}
\quad
\href{https://www.youtube.com/watch?v=Q3NYDF4JyTg}{Русскоязыное видео для сравнения.}

Название видео: \textbf{Как прожить невероятную жизнь}
	
Перевод субтитра на 08:10: \textbf{Тайир на тамильском означает йогурт}


\begin{wrapfigure}{r}{0.3\textwidth}
  \begin{center}
    \includegraphics[width=0.28\textwidth]{thumbnail}
  \end{center}
\end{wrapfigure}
PS: Справа~--- thumbnail. Его делает команда дизайнеров, не монтажер.
Хотя, если интересно — напишите Александру.



\tikz[remember picture,overlay] \node[opacity=0.15,inner sep=0pt] at (current page.center){\includegraphics[width=0.2\paperwidth]{IshaLogo}};
\end{document}
% https://www.youtube.com/watch?v=4UH7lzptFH8
