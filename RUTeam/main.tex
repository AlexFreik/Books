%&-job-name=newfilenameialwayswanted
\documentclass[
a4paper, % Stock and paper size.
12pt, % Type size.
article,
% oneside, 
onecolumn, % Only one column of text on a page.
% openright, % Each chapter will start on a recto page.
% openleft, % Each chapter will start on a verso page.
openany, % A chapter may start on either a recto or verso page.
]{memoir}
\synctex=1
\maxtocdepth{subsection}
\setsecnumdepth{subsection}
\counterwithout{section}{chapter}
%%% PACKAGES 
%%%------------------------------------------------------------------------------
\input{../preamble/rusBabel}
%%% Figures and colors
\usepackage[usenames,dvipsnames,svgnames,table,rgb]{xcolor}
\usepackage{tikz} % Figures
\usepackage{graphicx}  % Include figures
\usepackage{wrapfig}
\graphicspath{{img/}{../img/}} 


%%% INTERNAL HYPERLINKS
%%%-------------------------------------------------------------------------------

\usepackage{hyperref}   % Internal hyperlinks
\newcommand{\linkcolor}{blue}
\newcommand{\citecolor}{blue}
\newcommand{\filecolor}{magenta}
\newcommand{\urlcolor}{NavyBlue}
\hypersetup{				% Гиперссылки
    pdfborder={0 0 0},      % No borders around internal hyperlinks
	unicode=true,           % русские буквы в раздела PDF\\
	pdfstartview=FitH,
	colorlinks=true,  % false: ссылки в рамках; true: цветные ссылки
	linkcolor=\linkcolor,         % внутренние ссылки
	citecolor=\citecolor,        % на библиографию
	filecolor=\filecolor,      % на файлы
	urlcolor=\urlcolor,      % на URL
    linkbordercolor=\linkcolor,  % hyperlink border will be red 
    pdftitle={Yoga Veera Kit},
    pdfpagemode=FullScreen,
    pdfauthor={I am the Author} % author
}

\let\oldhref\href
\renewcommand{\href}[2]{\oldhref{#1}{\underline{#2}}}

   
\graphicspath{{img/}{../img/}{../FreqImg/}}


%%% PAGE LAYOUT 
%%%------------------------------------------------------------------------------

\setlrmarginsandblock{0.15\paperwidth}{*}{1} % Left and right margin
\setulmarginsandblock{0.10\paperwidth}{*}{1}  % Upper and lower margin
\checkandfixthelayout
%%% indenting
\setlength{\parindent}{0em}
\setlength{\parskip}{0.5em}
\begin{document}
\begin{center}
    \Huge \textbf{RU Team Guide}
\end{center}
\tableofcontents

\tikz[remember picture,overlay] \node[opacity=0.9,inner sep=0pt] at (page cs:0.8,0.8){\includegraphics[width=0.1\paperwidth]{IshaLogo}};

\subsection*{Предисловие}

Если вы читаете этот гайд с целью начать монтировать видео для YouTube~--- 
читайте только секцию \ref{montageRules} (со страницы \pageref{montageRules}). 





\newpage
\section{Фильтрация спама от AitTable}
\subsection{Gmail}
\begin{enumerate}
    \item Откройте любое \emph{нежелательное} письмо.
	    \begin{center} 
	        \includegraphics[width=0.9\textwidth]{AirTableSpam/gmail0} 
	    \end{center}
    \item Нажмите значек \textbf{"More"} и выберите 
                \textbf{"Filter messages like these"}.
	    \begin{center} 
	        \includegraphics[width=0.5\textwidth]{AirTableSpam/gmail1} 
	    \end{center}
    \item В открывшемся окне настройте параметры, как показано ниже. 
        Далее нажмите \textbf{"Create filter"}.
        \begin{itemize}
            \item From: \textbf{"noreply@airtable.com"}
            \item Subject: \textbf{"New comment on "}
        \end{itemize}

	    \begin{center} 
	        \includegraphics[width=0.9\textwidth]{AirTableSpam/gmail2} 
	    \end{center}
    \item Отметьте параметры \textbf{"Skip the Inbox (Archive it)"} и
        \textbf{"Mark as read"}, как показано ниже. 
        \begin{itemize}
            \item Далее все письма от \textbf{"noreply@airtable.com"}, 
                предмет которых которых содержит "New comment on "
                будут автоматически помечаться прочитанными и 
                перемещаться в папку \textbf{"Archive"}.
            \item Если отметить калочкой \textbf{"Also apply filter to ..."},
                то фильтр применится и к \emph{нежелательным} письмам, 
                которые уже пришли.
        \end{itemize}

	    \begin{center} 
	        \includegraphics[width=0.7\textwidth]{AirTableSpam/gmail3} 
	    \end{center}
\end{enumerate}


\newpage
\subsection{Yandex mail}
\begin{enumerate}
    \item Откройте любое \emph{нежелательное} письмо.
	    \begin{center} 
	        \includegraphics[width=0.9\textwidth]{AirTableSpam/ya0} 
	    \end{center}
    \item Нажмите значек \textbf{"More"} и выберите 
                \textbf{"Create filter"}.
	    \begin{center} 
	        \includegraphics[width=0.5\textwidth]{AirTableSpam/ya1} 
	    \end{center}
    \item В открывшемся окне настройте параметры, как показано ниже. 
        Далее нажмите \textbf{"Create filter"}.
        \begin{itemize}
            \item Не забудьте поменять \textbf{matches} на 
                \textbf{contains} в условии 
                фильтра по предмету письма и заменить текст на 
                \textbf{"New comment on "}.
            \item Если отметить калочкой \textbf{"Apply to existing messages"},
                то фильтр применится и к нежелательным письмам, которые 
                уже пришли.
            \item Далее все письма от \textbf{"noreply@airtable.com"}, предмет
                которых которых содержит \textbf{"New comment on "} 
                будут автоматически помечаться прочитанными 
                и перемещаться в папку \textbf{"Archive"}.
        \end{itemize}

	    \begin{center} 
	        \includegraphics[width=0.9\textwidth]{AirTableSpam/ya2} 
	    \end{center}

\end{enumerate}

\newpage
\section{Монтаж}\label{montageRules}
\subsection{Правила для YouTube}
В правилах мы выделили наилучшие практики, которые позволяют выдерживать качество и сохранять общий стиль.
\begin{enumerate}
\item Параметры экспорта:
    \begin{itemize}
        \item \textbf{кодек --- H264};
        \item \textbf{битрейт видео~--- 10mbps};

        Если ваш интернет не позволяет 10mbps можно взять битрейт чуть выше, чем в 
        оригинальном видео (обычно где-то 2-3 mbps).

        \item \textbf{битрейт аудио --- 320};
        \item \textbf{разрешение --- $1920 \times 1080$};
       
        Если разрешение оригинала $1280 \times 720$,
            то фиальное видео делаем в формате
        $1920 \times 1080$ {\color{gray}(рекомендательные алгоритмы 
            YouTube поощряют 
        высокоформатные видео)}. 

        Но, если качесво $480$\ или ниже --- то оставляем 
        качесво оригинала {\color{gray}(растягивать $480$ до $1080$ уже перебор)}.
    \end{itemize}



\item У квадратных видео дублируем его-же на бекграунд, растягивем до границ, и накладываем блюр.

	\begin{center} \includegraphics[width=0.5\textwidth]{tooWide}  \end{center}

\item Устаревшие заставки <<Sadhguru. Yogi, mystic and visioner.>> и <<Conversation with Mystic>> вырезаются. {\color{gray}Они только отнимает время.}

\item Конечный слайд нужно заменять на \href{https://drive.google.com/file/d/11NbSgvq8LbxDcy-a2WY5OJTKUZKcZx88/view?usp=sharing}{русский}. 

	Конечная надпись <<\textcopyright\ Sadhguru 2021>> не меняется.

 Часто используемые слайды на 
\href{https://drive.google.com/drive/folders/1O54z3DtKpl90ut0Aa8wYkEEP37e00zPY?usp=sharing}{Google Drive}.

\item Вставки с текстом (например, вопросов). Если текст не озвучивается, то время фрагмента с текстом должно быть достаточным, чтобы вы могли неспеша прочитать его (возможно, придется удлиннить видео).

\item Иногда в английской версии вставляют субтитры прямо в видео
    {\color{gray}(обычно, во время неразборчивого вопроса)}. Нам их переводить
    \textbf{не} нужно, поскольку русская озвучка четкая, 
    и субтитры всегда можно включить.

    Но, в случае, когда это не субтитры, а слайд с вопросом~--- 
    делать русский слайд нужно, даже если вопрос озвучивают.
\item Английский текст при переводе не должен быть виден. Для этого порой лучше
    накрывать переводом с запасом по времени слева и справа. То есть, резать
    не точно по началу и концу, а с отступами.

    В случае сложной анимации, этим можно пренебречь в угоду общей эстетики.

\item Любой сколько-нибудь сложный перевод, особенно если есть какие-то названия, нужно согласовать с отвественным человеком (Лара Полевая или Ксения Ошерова).

	Перевод вопросов и т.д. обычно имеется у звукоря.

\item Дизайнерский отдел Иши дал следующие шрифты: Merriweather для заголовков, крупного текста и Open Sans для субтитров. Используем их.

    Иногда используется Segoe Script, вы его сразу заметите.

 \includegraphics[width=0.3\textwidth]{segoeScript}

    Их можно скачать и установить из Google Fonts:
    \begin{itemize}
        \item  Merriweather \href{https://fonts.google.com/specimen/Merriweather}{\small https://fonts.google.com/specimen/Merriweather};
        \item Open Sans \href{https://fonts.google.com/specimen/Open+Sans}{\small https://fonts.google.com/specimen/Open+Sans};
    \item Segoe Script \href{https://www.fonts.com/font/microsoft-corporation/segoe-script?QueryFontType=Web&src=GoogleWebFonts}{\small https://www.fonts.com/font/microsoft-corporation/segoe-script?QueryFontType=Web\&src=GoogleWebFonts}.
      \end{itemize}

      \emph{PS}: Устанавливать шрифты очень легко, ниже (субсекция \ref{fonts}) можно
      найти инструкцию для MacOS.


\item Переходы между фрагментами с разным фоном через Dip to Black, иначе через Dissolve.


\item Перевод / добавление субтитров. По ситуации~--- обычно блюр красивее непрозрачного фона, но иногда нет. 

\item Текст без особого смысла, вроде "Darshan — Dec 2012
    Isha Yoga Center" оставляем как есть, наше время ценнее =). 
    {\color{gray}Плюс, не 
   встречал достаточно гармоничной замены, чтобы это оправдывало добаление
   не важной инфы.}
\end{enumerate}

\subsection{Советы}
\begin{enumerate}
\item \textbf{Основные методы перевода текста.}

    \begin{itemize}
        \item \emph{Наложить текст с непрозрачным бекграундом.} 

            \begin{itemize}
                \item Подходит для ситуации, когда фон в месте, где размещен текст
            одноцветен, но обычно это не так, и лучше использовать другие методы.
            \end{itemize}

        \item \emph{Наложить gaussian blur на место где находится текст, и добавить
            свой поверх.}

            \begin{itemize}
                \item Универсальный способ, подходит для 
            динамического бекграунда, и, обычно, смотрится неплохо.

                \item Ниже пример, где лучше было использовать блюр.

\begin{figure}[!hbp]
\centering
\begin{minipage}[b]{0.4\textwidth}
\includegraphics[width=\textwidth]{titleBlur}
\caption{Nice.}
\end{minipage}
\hfill
\begin{minipage}[b]{0.4\textwidth}
\includegraphics[width=\textwidth]{titleBlurBad}
\caption{Not nice.}
\end{minipage}
\end{figure}
                \item Если фон сливается с переведенным текстом, 
                    или заблюренный английскй текст 
                    добавляет слишком много белого/ черного на фон, то 
                    можно добавить полупрозрачный бекграунд на переведенный текст.
            \end{itemize}

           \item \emph{Фиксация кадра, когда английский текст еще не появился,
               и добавить свой текст поверх.}

            \begin{itemize}
                \item Подходит в случае статической картинки на бэкграунде. Если 
                на заднике красивая анимация~--- blur предпочтительнее.
               
            \item Если в оригинале естть zoom (динамическое увеличение
                статической картинки) для добавления динамики, то нужно
                добавить zoom и при переводе.

        \end{itemize}
    \end{itemize}
\item Чтобы скачать видео с YouTube можно воспользоваться сайтом
    \href{https://www.y2mate.com/}{https://www.y2mate.com/}, скопировав и вставив 
    ссылку на видео. 

    Также, если видео уже открыто в браузере можно просто вставить "pp" после
    слова youtube в ссылке, и y2mate откроется автоматически.

    Например, https://www.youtube.com/watch?v=Oi7eLmaL1DU нужно преобразовать
    в https://www.youtube\textbf{pp}.com/watch?v=Oi7eLmaL1DU .
\item При выборе приложения для монтажа, DaVinci Resolve~---
    отличный вариант с бесплатной версией, которой более чем хватает. 

В ашраме в основном используется Adobe Premiere Pro, насколько знаю. 
Чтобы его спиратить, наберите в Youtube "Premiere Pro download
MacOS/Windows free" и выберите видео с большим числом просмотров и положительными отзывами.

\emph{PS}: Этот волонтер имел опыт как с Premiere, так и DaVinci, и
сейчас, в основном, использует DaVinci.
\item Для пользователей MacOS самый простой вариант монтирующей системы~--- 
    iMovie, но у нее есть три серьезных недостатка.
    \begin{itemize}
    \item Нет возможности добавить текст поверх видео. То есть, придется делать
        картинку с перевдодм в отдельном приложении и наклеивать поверх.
        Это нарушает поток, и нужно будет повозиться, если придется исправлять текст.
    \item Можно использовать только два видео трека. Это ограничивает более 
        сложных монтаж, хотя обычно двух хватает.
    \item Нет возможности почеловечески кастомизировать разрешение и формат видео. 
\end{itemize}
    Поэтому совет~--- скачать более профессиональное приложение, тем более что
    DaVinci бесплатный =).
\end{enumerate}





\subsection{Обучающие материалы}

Монтировать совсем не сложно! В начале будет немножко трудно, но основы можно 
освоить за пару часов. Главное~--- не бояться, и спрашивать, если что-то не понятно.

Так же, Google~--- ценный помошник, особенно, если гуглить на английском.
Например, запросы, вроде 
"DaVinci how to add frame hold" ("DaVinci как зафиксировать фрейм видео")
должны помочь.

Ниже real-life примеров монтажа, можете ознакомится с ними!
\begin{itemize}
    \item Подробное видео-пример монтажа видео в DaVinci. 
        Пример для ролика с YouTube, но, в 
        Instagram примерно те-же задачи)
        
        DaVinci exmple:~--- \href{https://youtu.be/SAceoBqdAvw}{https://youtu.be/SAceoBqdAvw}
    \item Короткое видео о том, как фиксировать фрейм и добавить zoom.

        DaVinci Frame Freez example:~--- \href{https://youtu.be/caPaZ5syTC8}{https://youtu.be/caPaZ5syTC8}
    \item Короткий туториал о том что делать со старыми "квадратными" видео. 

        DaVinci Narrow Video Tutorial:~--- \href{https://youtu.be/FGdErSdSOAI}{https://youtu.be/FGdErSdSOAI}

\item Три двух-минутных ролика монтажа для Instagram в Premiere Pro.

    Premiere short examples:
    \begin{itemize}
        \item \href{https://youtu.be/AWOsM9fX9RQ}{https://youtu.be/AWOsM9fX9RQ}
        \item \href{https://youtu.be/lvRnpvTXHis}{https://youtu.be/lvRnpvTXHis}
        \item \href{https://youtu.be/Ontq3mMD9AM}{https://youtu.be/Ontq3mMD9AM}
    \end{itemize}
\end{itemize}


\subsection{Test Task (Тренировочное видео) YouTube}
 
Задание~--- смонтировать видео и прислать его тому кто его дал. 

Загрузите видео напрямую в Telegram, то есть не нужно 
загружать на Google Drive и присылать ссылку)

\begin{center} \textbf{Видео} \end{center}
\href{https://www.youtube.com/watch?v=9sGJUR7stzc}{Оригинал.}
\quad
\href{https://drive.google.com/file/d/1Y6ECjMSvkaUFmNawIePfFvqS2ZnB3SPi/view?usp=sharing}{Аудио трек.}
\quad
\href{https://www.youtube.com/watch?v=Q3NYDF4JyTg}{Русскоязыное видео для сравнения.}

Название видео: \textbf{Как прожить невероятную жизнь}
	
Перевод субтитра на 08:10: \textbf{Тайир на тамильском означает йогрт}


\begin{wrapfigure}{r}{0.3\textwidth}
  \begin{center}
    \includegraphics[width=0.28\textwidth]{thumbnail}
  \end{center}
\end{wrapfigure}

\emph{PS}: На русском канале довольно уродливый субтитр, призываю вас улучшить его =).

\emph{PP}S: Справа~--- thumbnail. Его делает команда дизайнеров, не монтажер.
Хотя, если интересно — напишите Александру.


\newpage
\subsection{Технические моменты}
\subsubsection{Установка шрифтов MacOs}\label{fonts}
\begin{enumerate}
    \item Скачайте шрифт, например, из Google Fonts.
    \item Откройте приложение \textbf{"font Book"}.
	    \begin{center} 
	        \includegraphics[width=0.9\textwidth]{fontsInstallation/macos0} 
	    \end{center}

    \item Нажмите на \textbf{"add fonts"}.
	    \begin{center} 
	        \includegraphics[width=0.5\textwidth]{fontsInstallation/macos1} 
	    \end{center}
    \item Выберети скаченный файл и нажмите на \textbf{"Open"}.
	    \begin{center} 
	        \includegraphics[width=0.9\textwidth]{fontsInstallation/macos2} 
	    \end{center}
\end{enumerate}



\newpage
\thispagestyle{empty}
\tikz[remember picture,overlay] \node[opacity=0.15,inner sep=0pt] at (current page.center){\includegraphics[width=0.2\paperwidth]{IshaLogo}};
\end{document}
% https://www.youtube.com/watch?v=4UH7lzptFH8
