%&-job-name=newfilenameialwayswanted
\documentclass[
a4paper, % Stock and paper size.
12pt, % Type size.
article,
% oneside, 
onecolumn, % Only one column of text on a page.
% openright, % Each chapter will start on a recto page.
% openleft, % Each chapter will start on a verso page.
openany, % A chapter may start on either a recto or verso page.
]{memoir}
\maxtocdepth{subsection}
\setsecnumdepth{subsection}
\counterwithout{section}{chapter}
%%% PACKAGES 
%%%------------------------------------------------------------------------------
\input{../preamble/rusBabel}
%%% Figures and colors
\usepackage[usenames,dvipsnames,svgnames,table,rgb]{xcolor}
\usepackage{tikz} % Figures
\usepackage{graphicx}  % Include figures
\usepackage{wrapfig}
\graphicspath{{img/}{../img/}} 


%%% INTERNAL HYPERLINKS
%%%-------------------------------------------------------------------------------

\usepackage{hyperref}   % Internal hyperlinks
\newcommand{\linkcolor}{blue}
\newcommand{\citecolor}{blue}
\newcommand{\filecolor}{magenta}
\newcommand{\urlcolor}{NavyBlue}
\hypersetup{				% Гиперссылки
    pdfborder={0 0 0},      % No borders around internal hyperlinks
	unicode=true,           % русские буквы в раздела PDF\\
	pdfstartview=FitH,
	colorlinks=true,  % false: ссылки в рамках; true: цветные ссылки
	linkcolor=\linkcolor,         % внутренние ссылки
	citecolor=\citecolor,        % на библиографию
	filecolor=\filecolor,      % на файлы
	urlcolor=\urlcolor,      % на URL
    linkbordercolor=\linkcolor,  % hyperlink border will be red 
    pdftitle={Yoga Veera Kit},
    pdfpagemode=FullScreen,
    pdfauthor={I am the Author} % author
}

\let\oldhref\href
\renewcommand{\href}[2]{\oldhref{#1}{\underline{#2}}}

   
\graphicspath{{img/}{../img/}{../FreqImg/}}


%%% PAGE LAYOUT 
%%%------------------------------------------------------------------------------

\setlrmarginsandblock{0.15\paperwidth}{*}{1} % Left and right margin
\setulmarginsandblock{0.10\paperwidth}{*}{1}  % Upper and lower margin
\checkandfixthelayout
%%% indenting
\setlength{\parindent}{0em}
\setlength{\parskip}{0.5em}
\begin{document}
\begin{center}
    \Huge \textbf{Русский гайд}
\end{center}
\tableofcontents

\tikz[remember picture,overlay] \node[opacity=0.9,inner sep=0pt] at (page cs:0.8,0.8){\includegraphics[width=0.1\paperwidth]{IshaLogo}};

\section{YouTube Videos Translation Process}

\subsection{\ldots}
\ldots
\subsection{Создание аудио трека}
Подробнее должен рассказать Эльдар.
\subsection{Создание финального видео}
Эльдар после создания аудио трека отмечает ответственног за этот этап в AirTable. В последнем можно будет найти ссылки на оригинальное видео (раздел <<URL ORIGINAL>>), дедлайн (<<GIVE TO EMEDIA BY>>) и аудио трек (<<MEDIA LINK>>). 

Так же там будут два поля с названиями: <<TN TITLE>> и  <<TITLE>>. {\color{gray}Разница в том, что первое используется для tumbneils (это картинка, которая будет показываться у этого видео в ленте, TN как раз сокращение от tumbneil). В то же время, <<TITLE>> используется для названия видеоролика. }

Монтажеру рекомендуется использовать <<TN TITLE>> при руссификации заставки, если это необходимо.

После того как видео смонтированно, \hyperref[montageRules]{подробнее об монтаже тут}, нужно загрузить получившийся продукт на гугл диск по ссылке <<MEDIA LINK>>, после чего 
\begin{itemize}
    \item отметить ответственного за следующий этап (Александра Гринева) в поле <<COLLABORATORS>>, 
\item в <<VIDEO STATUS>> отметить <<Video ready>>.
\end{itemize}

В случае, если видео исправляется и перезаливается необходимо сообщить об этом ответственному за следующий этап (Александру Гриневу), чтобы новая версия так же была проверенна.

\subsection{\ldots}
\ldots

\section{Монтаж --- советы \& правила}\label{montageRules}
\subsection{Правила}
\begin{enumerate}
\item Параметры экспорта:
    \begin{itemize}
    \item кодек --- H264;
    \item битрейт видео~--- 8mbps;
    \item битрейт аудио --- 320;
    \item разрешение --- $1920 \times 1080$;
    \end{itemize}

    Важно отметить, что даже если видео было в качестве $240 \times 240$ (то есть, еще и в другом формате), то нужно делать разрешение $1280 \times 720$.
\item Если в начале есть заставка <<Sadhguru. Yogi, mystic and visioner.>>, то ее нужно просто отрезать, она только занимает время, в новых видео ее больше нет. 

    Это же касается и других устаревших заставок, вроде <<Conversation with Mystic>>.
\item Конечный слайд нужно заменять на русский. 

    Так же в новых видео в самом конце есть надпись <<\textcopyright\ Sadhguru 2021>>, ее оставляем без изменений. {\color{red} add info}

 Часто используемые слайды без текста (и с переведенным текстом) можно найти на \href{https://drive.google.com/drive/folders/1moOa4wR201aplGnueFKTXclzGQBKhxzc?usp=sharing}{Google Drive}.

\item У некоторых видео есть вставки с текстом (например, вопросов). В таких случаях, если текст не озвучивается, то время фрагмента с текстом нужно самостоятельно подстроить под русскую версию, так чтобы зритель успевал прочитать весть текст, не ставя на паузу (то есть удлинить или укоротить, если требуется).  

\item В идеале, если английский текст заменятся русским, то нужно смонтировать так, чтобы его было совсесм не видно. Но в случаях, когда на видео присутствует сложная анимация, этим правилом можно пренебречь в угоду общей эстетики.

\item Любой сколько-нибудь сложный перевод, особенно если есть какие-то названия, лучше согласовать с отвественным человеком. Сейчас это~--- Юрий Кузмин.
\item Есть два шрифта, которые нам официально рекомендовал дизайнерский отдел Иши: Merriweather для заголовков, крупного текста и Open Sans для субтитров.

    Так же иногда используется Segoe Script (например, на слайдах перед началом практики, врде Саштанги), вы его сразу заметите).

    Их можно скачать и установить из Google Fonts:
    \begin{itemize}
        \item  Merriweather \href{https://fonts.google.com/specimen/Merriweather}{\small https://fonts.google.com/specimen/Merriweather};
        \item Open Sans \href{https://fonts.google.com/specimen/Open+Sans}{\small https://fonts.google.com/specimen/Open+Sans};
    \item Segoe Script \href{https://www.fonts.com/font/microsoft-corporation/segoe-script?QueryFontType=Web&src=GoogleWebFonts}{\small https://www.fonts.com/font/microsoft-corporation/segoe-script?QueryFontType=Web\&src=GoogleWebFonts}.
      \end{itemize}
\end{enumerate}

\subsection{Советы}
\begin{enumerate}
\item В видео иногда будет текст, который не несет особого смысла, вроде небольшого в левом нижнем углу вида "Darshan — Dec 2012
    Isha Yoga Center", его можно оставить как есть, ваше время ценнее =).

\item Если видео более квадратное чем нужно (встречается у старых видео), то можно на бекграунд добавить это-же видео, растынув его до границ, и наложив блюр. Так по краям окажутся не просто черные полосы, а что-то более очаровательное. 


\item Для пользователей MacOS самый простой вариант монтирующей системы~--- iMovie, но у нее есть три серьезных недостатка.
    \begin{itemize}
    \item Самое неприятное --- то что нет возможности добавить текст поверх видео. То есть для того чтобы заменить текст на русский придется делать картинку в отдельном приложении, и потом эту картинку вставлять в видео поверх. Это довольно неприятно, так как замедляет процесс, не говоря о том, если придется исправлять текст, то нудно будет сделать все то же самое по второму кругу.
    \item Можно использовать не более двух видео треков. Это сильно ограничивает в действиях, если нужен сколько-нибудь сложных монтаж, хотя обычно двух треков хватает.
    \item Нет возможности почеловечески кастомизировать разрешение и формат видео. 
\end{itemize}
Поэтому совет~--- скачать более профессиональное приложение. В ашраме в основном используется Adobe Premiere Pro, насколько я слышал.
\end{enumerate}

\tikz[remember picture,overlay] \node[opacity=0.15,inner sep=0pt] at (current page.center){\includegraphics[width=0.2\paperwidth]{IshaLogo}};

\end{document}
% https://www.youtube.com/watch?v=4UH7lzptFH8
