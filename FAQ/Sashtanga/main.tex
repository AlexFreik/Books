\documentclass[
a4paper, % Stock and paper size.
12pt, % Type size.
article,
% oneside, 
onecolumn, % Only one column of text on a page.
% openright, % Each chapter will start on a recto page.
% openleft, % Each chapter will start on a verso page.
openany, % A chapter may start on either a recto or verso page.
]{memoir}

%%% PACKAGES 
%%%------------------------------------------------------------------------------

\usepackage[utf8]{inputenc} % If utf8 encoding
% \usepackage[lantin1]{inputenc} % If not utf8 encoding, then this is probably the way to go
\usepackage[T1]{fontenc}    %
\usepackage[english,russian]{babel} % English please
\usepackage[final]{microtype} % Less badboxes

% \usepackage{kpfonts} %Font

\usepackage{amsmath,amssymb,mathtools} % Math

%%% Figures and colors
\usepackage[usenames,dvipsnames,svgnames,table,rgb]{xcolor}
\usepackage{tikz} % Figures
\usepackage{graphicx}  % Include figures
\usepackage{wrapfig}
\graphicspath{{img/}{../img/}}

%%% PAGE LAYOUT 
%%%------------------------------------------------------------------------------

\setlrmarginsandblock{0.08\paperwidth}{*}{1} % Left and right margin
\setulmarginsandblock{0.15\paperwidth}{*}{1}  % Upper and lower margin
\checkandfixthelayout
%%% indenting
\setlength{\parindent}{0em}
\setlength{\parskip}{0.5em}

%%% INTERNAL HYPERLINKS
%%%-------------------------------------------------------------------------------

\usepackage{hyperref}   % Internal hyperlinks
\hypersetup{
pdfborder={0 0 0},      % No borders around internal hyperlinks
pdfauthor={I am the Author} % author
}
% ----------- hyper ref -----------
\usepackage{hyperref}


\newcommand{\linkcolor}{blue}
\newcommand{\citecolor}{blue}
\newcommand{\filecolor}{magenta}
\newcommand{\urlcolor}{NavyBlue}
\hypersetup{				% Гиперссылки
	unicode=true,           % русские буквы в раздела PDF\\
	pdfstartview=FitH,
	colorlinks=true,  % false: ссылки в рамках; true: цветные ссылки
	linkcolor=\linkcolor,         % внутренние ссылки
	citecolor=\citecolor,        % на библиографию
	filecolor=\filecolor,      % на файлы
	urlcolor=\urlcolor,      % на URL
}
\usepackage{memhfixc}   %






\begin{document}
\chapter*{Часто задаваемые вопросы}
\tikz[remember picture,overlay] \node[opacity=0.15,inner sep=0pt] at (current page.center){\includegraphics[width=0.5\paperwidth,height=0.5\paperheight]{IshaLogo}};
\begin{enumerate}
    \item \textbf{Кто может заниматься этой практикой?}

За исключением беременных женщин, заниматься этой практикой может любой, в том
числе женщины во время менструального цикла, а также те, кто страдает
хроническими заболеваниями или любыми другими заболеваниями (такими как астма,
мигрень, диабет, высокое кровяное давление, болезни сердца, заболевания легких,
глаукома, катаракта, отслойка сетчатки, грыжа и прочее.)

Люди с травмами спины, шеи или колена также могут выполнять эту практику с
осторожностью, насколько это возможно.
\item \textbf{Могут ли беременные женщины выполнять практику?}

Нет, беременным женщинам не следует заниматься этой практикой.
\item \textbf{С какого минимального и максимального возраста можно выполнять эту практику?}

Ограничений нет. Этой практикой можно заниматься в любом возрасте.
\item \textbf{Как долго следует воздерживаться от этой практики после операции?}

Воздержитесь в течении 6 месяцев после серьезной операции и в течении 6 недель
после небольшой операции.


\item \textbf{Как долго мне нужно ждать после еды, чтобы выполнить эту практику?}

Практика должна выполняться натощак.
Это означает, что должно пройти не менее:
\begin{itemize}
\item 4 часов после полноценного приема пищи
\item 2,5 часа после перекуса, например такого, как фрукт или несколько печений
\item 1,5 часа после напитка, такого как чай или кофе
\item Воду пить можно
\end{itemize}

\item \textbf{Могу ли я есть сразу после практики?}

Да. Вы можете съесть или выпить напиток комнатной температуры сразу после
практики. Подождите 10-15 минут, прежде чем употреблять что-нибудь охлажденное.
\item \textbf{Могу ли я принять душ сразу после практики?}

После практики подождите 15-20 минут, прежде чем принимать горячий душ, и 25-30
минут, прежде чем принимать холодный душ.

\item \textbf{Сколько раз в день я могу выполнять эту практику?}

Вы можете делать 3 цикла подряд 4-5 раз в день.
\item \textbf{Могу ли я самостоятельно обучать этой практике, если у человека нет возможности посмотреть видео?}

Нет. Практика основана на очень тонкой науке. Если йогические практики выполняются
правильно, они способны трансформировать вашу жизнь, поэтому обычно требуются
годы обучения, чтобы гарантировать, что они передаются правильным образом. Вы
можете использовать видео как средство, чтобы предложить практику другим. Если вы
заинтересованы в обучении этой и другим практикам, свяжитесь с нами, чтобы узнать
больше о программе подготовки преподавателей.
\item \textbf{Как выполнять эту практику с другими практиками Иши? Есть ли какая-то особая последовательность?}

Нет, такой последовательности нет.
\item \textbf{Что делать, если не получается правильно принять позу?}
(Например, нос или живот касается пола или не получается удерживать позу в течение
6-7 минут подряд)

Ничего страшного, постарайтесь удерживать позу настолько, насколько это для вас
возможно. Если вы будете практиковать регулярно, то ваше тело разовьет
необходимую силу, и вы постепенно сможете принять идеальную позу.

\item \textbf{Что делать, если я не могу удерживать позу 6-7 минут подряд?}

Если вы не можете удерживать позу 6-7 минут, вы можете начать с 3 минут и
постепенно наращивать время до 6-7 минут.


\tikz[remember picture,overlay] \node[opacity=0.15,inner sep=0pt] at (current page.center){\includegraphics[width=0.5\paperwidth,height=0.5\paperheight]{IshaLogo}};



\item \textbf{Как мне определять длительность каждой части: 6-7 минут для саштанги, 3-4 минуты для макарасаны? Могу ли я установить таймер на моем телефоне?}

Очень важно не устанавливать таймеры или будильники. Это внутренний процесс.
Любой вид оповещения отвлечет ваше внимание вовне. Ориентируйтесь на свои ощущения времени. Первоначально вы можете проверять свои часы между каждым этапом практики, прежде чем переходить к следующему этапу.

Например, проверьте свои часы, прежде чем начинать Саштангу. После выхода из Саштанги, если время составляет 6-7 минут, переходите к следующему этапу. Если это меньше 6 минут, удерживайте Саштангу еще некоторое время.

Постепенно вы сможете отслеживать время, опираясь на свои ощущения. Если вы
затратите немного больше или меньше времени, то ничего страшного.

\item \textbf{Удерживая Саштангу, у меня проскальзывают лоб и пальцы ног. На какой
    поверхности ее лучше делать?}

Поскольку Саштанга вытягивает позвоночник и шею, в то время, как живот поднят, вы
можете обнаружить, что ваш лоб или пальцы ног могут скользить, особенно если вы
практикуете на гладкой поверхности. Вы можете практиковаться на коврике для йоги
или найти поверхность, которая лучше позволяет контролировать позу и поможет вам
ее удерживать.

По мере того, как вы продолжите практиковать и приобретете гибкость, вы увидите, что
удерживать позу станет легче и удобнее.
\item \textbf{Как узнать, дышу ли я немного глубже, чем обычно?}

Какую бы позу вы ни приняли, вы заметите, что ваше дыхание изменилось. Как только
вы войдете в Саштангу, обратите внимание на то, как вы обычно дышите в этой позе, и
вам следует дышать немного глубже, чем обычно.

\item \textbf{Я чувствую, что не могу дышать в этой позе. Я делаю что-то неправильно?}

Ваше дыхание не будет таким глубоким и комфортным, как в обычной вертикальной
или сидячей позе. Оно может казаться поверхностным, и это нормально. Если вам
слишком неудобно, вы можете удерживать эту позу насколько это возможно для вас.

Если вам трудно дышать, вы можете немного расслабить голову, стараясь держать нос
как можно выше над землей.
Приняв позу, убедитесь, что ваше тело расслаблено, и вы дышите немного глубже, чем
обычно.
Если вам по-прежнему трудно дышать, вы можете немного опустить живот, насколько
возможно не касаясь земли.

\end{enumerate}

Если у вас остались вопросы по практике, напишите нам по адресу \\ 
sadhanasupport.russian@ishafoundation.org или \href{https://isha.sadhguru.org/in/en/yoga-meditation/yoga-teacher-training/hatha-yoga-teacher-training/teachers-in-your-area}{свяжитесь с преподавателем хатха-йоги}.
\tikz[remember picture,overlay] \node[opacity=0.15,inner sep=0pt] at (current page.center){\includegraphics[width=0.5\paperwidth,height=0.5\paperheight]{IshaLogo}};

\end{document}
