\chapter{Введение}

Друзья, больше записей о кулинарии не будет, я отдал всё. У меня была потребность поделиться опытом, и я считаю, что я восстановил баланс. Буду рад, если кто-то начнёт внедрять знания в жизнь, а не просто ставить сердечки под записями. 

Несколько рекомендаций для тех, кто думает о здоровье:
\begin{itemize}
%\item Пейте чистую воду за 15 минут перед трапезой, и в течение дня чем больше, тем лучше. Чистая вода — это дистиллят или близко к этому.
\item Ешьте мёд вместо сахара и просто так.
\item Пейте свежевыжатые соки или просто взбитые цитрусы по утрам.
\item  Избавляйтесь от неестественных продуктов (медикаментов, химии, алкоголя, консервантов, мяса, рыбы, молочки, сахара, соли и хлеба).
\item Пейте зелёные коктейли
\item  Ешьте свежие плоды — нашу видовую пищу.
\item Готовьте с позитивным настроем и любимой музыкой.
\item Голодайте раз в неделю или две, дайте телу отдых.
\item  Не сублимируйте и не делайте из еды культа — найдите любимое дело.
\item Не путайте голод и жажду. 
%Согласно Аюрведе нужно есть только в состоянии голода, если это легкий аппетит на что-то конкретное, например, вкусный снек, то это нужда ума, а не тела
\item Самый сильный огонь пищеварения приходится на полдень. Планируйте основную трапезу на это время.
\item Не тратьте деньги на дорогие и экзотические продукты — всё самое необходимое растёт в нашей полосе.
\item  Прислушивайтесь к себе, только вы знаете что для вас лучше в данный момент.
\item Психологическое здоровье важнее телесного, незачем себя ломать.
\item Дайте телу физические упражнения.
\item Упрощайте сложные операции
\item Здоровье — это ежедневный труд. Сделайте так, чтобы этот труд был в радость.
\end{itemize}

\begin{quote}
    \centering
Посеешь мысль — пожнёшь поступок,\\
посеешь поступок — пожнёшь привычку,\\
посеешь привычку — пожнёшь характер,\\
посеешь характер — пожнёшь судьбу\ldots 
\end{quote}
