\chapter{Советы \& лайфхаки}

\section{Почему все так просто}

В моей подборке нет таких блюд, как щи или картошка, потому что я считаю их абсолютно пустыми, также нету блюд, которые готовятся по несколько часов, в несколько этапов, во фритюре и т.д. Я призываю вас уходить от подобной кулинарной глупости.

Да, я могу испечь печенье для мамы, и только потому, что она просит и я её очень люблю. Но это отнимает уйму времени и грязной посуды. Если вы мужчина и решили приготовить какие-нибудь пирожки, селедку под шубой или фалафель, то скорее всего либо выдохнетесь, либо вообще разочаруетесь и лучше купите эту чертову селедку в ближайшем магазе и забудете про кулинарию надолго. Если же вы женщина, то муж вам после такого угощения пожизненно обязан. Цените своё время!!!

Мне всегда больно смотреть на людей, которые тратят часы на приготовление блюда, которое того не стоит, просто не стоит. Подумайте: блинчики, пирожки и булочки, сложные салаты, фритюр, бурфи, жареный сыр, даже плов, жюльен и халава — это всё бесполезная трата времени, вред для здоровья и издевательство над едой. Ведь в итоге, как бы вы не старались, это окажется внутри, а балласт выйдет известным путём. Не становитесь заложниками времени и жертвами блюдомании. Оцените трезво, что действительно нужно вашему телу, а что навязано извне.

Поэтому\ldots

Я подобрал для вас полезные, простые и функциональные блюда, из которых можно составить полноценное меню на неделю. Описание сокращено до минимума, чтобы рецепт можно было за 2 секунды окинуть взглядом и понять принцип процесса.

\section{Про самообразование}

А теперь напишу о том, как сделать всё ещё проще и в радость. Некоторые советы для начинающих, чтобы популярные вопросы отпали сразу:
\begin{itemize}
\item  Купите хорошую доску, набор удобных для вас ножей, посуды и других инструментов, чтобы приготовление еды приносило радость, и не превращалось в каторгу. Вам скорее всего понадобятся: небольшой казан, посуда нержавейка с толстым дном 2х-3х объёмов, лопатка, глубокое сито с ушками + миска под него, половник, большая ложка, большой широкий нож, малый нож, картофелечистка, тёрка, электрический чайник, стационарный и погружной блендер с мельничкой, кофемолка, духовка, плита и силиконовый противень.
\item  Установите обратноосмотический фильтр, забудьте про накипь в чайнике и наслаждайтесь чистой водой.
\item  Помните, что газовая плита всегда лучше конвекторной, а живой огонь лучше газа.
\item  Поинтересуйтесь в интернете о том, как чистить привычные вам продукты и вы узнаете много нового. Хороший пример: арбуз, ананас, авокадо или гранат. Возможно я даже сделаю подборку видео на тему.
\item  Научитесь эффективно и быстро резать. Это сэкономит вам тысячи часов вашей жизни.
\item  Режьте продукты так, чтобы: они сохраняли вкус, помещались в ложку и быстро готовились.
\item  Если падает нож, не спешите его ловить.
\item  Храните крупы и специи, свежевыжатые соки, супы и приготовленные блюда герметично в контейнерах.
\item  Узнайте вообще, как и где лучше хранить разные продукты.
\item  Узнайте время варки продуктов.
\item  Пробуйте и нюхайте каждый продукт перед закладкой. Помните о сроках годности.
\item  Мойте крупу перед варкой.
\item  В сладкие блюда добавляйте щепотку соли, а в солёные немного сахара.
\item  Помните, что мука — это клей, а сахар — яд.
\item  Всегда ищите лучшие пути для выполнения привычных операций.
\end{itemize}

\begin{quote}
    \emph{``Для многих кулинария — это творчество. Но в мире творчества есть тысячи гораздо более интересных занятий для души. Помните, что это всего лишь пища и её главная задача — питать наше тело.''}

\end{quote}

\section{Про растительные масла}

Про вред рафинированных исписан весь интернет, читайте сами. Здесь про нерафинированные холодного отжима

Если рассматривать растительные масла как функциональный продукт, то по действию они все примерно одинаковы — помогают выводить жирорастворимую слизь со стенок кишечника, и различаются только ароматом. 

Для заправки овощных салатов (под овощами я имею ввиду корнеплоды и зелёную массу) можно брать самое простое подсолнечное холодного отжима. Нормальное масло отжимается на деревянном прессе, не нагревается и практически не пахнет, в отличие от дешёвых нерафинированных. Прийдётся поискать, но оно того стоит.

А для удовольствия можете брать любое по душе и кошельку: кедровое, кунжутное, грецкого ореха, тыквенное — просто для аромата.
Не спорю, есть и лечебные масла, как облепиховое, черного тмина, и прочие, но их пользу я не рассматриваю, потому что считаю костылями, которые при неадекватном питании не помогут.

Раз уж разговор зашёл об очищении, вот вам салатик для обнуления кармы:
\begin{itemize}

\item морковь и китайская капуста 500
\item кориандр 1 чл
\item чили 2-4 щепотки
\item мёд 2 чл
\item лимонный сок 20 мл
\item зелень 2 горсти
\item масло растительное х/о 2-4 стл
\end{itemize}

Корнеплоды на крупной тёрке и заправить специями. Хорошая метла для тех, кто давно не чистился. 


Здравия!

