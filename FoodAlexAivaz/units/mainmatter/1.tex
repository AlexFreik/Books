\chapter{Советы \& лайфхаки}

\section{Почему все так просто}

В моей подборке нет таких блюд, как щи или картошка, потому что я считаю их абсолютно пустыми, также нету блюд, которые готовятся по несколько часов, в несколько этапов, во фритюре и т.д. Я призываю вас уходить от подобной кулинарной глупости.

Да, я могу испечь печенье для мамы, и только потому, что она просит и я её очень люблю. Но это отнимает уйму времени и грязной посуды. Если вы мужчина и решили приготовить какие-нибудь пирожки, селедку под шубой или фалафель, то скорее всего либо выдохнетесь, либо вообще разочаруетесь и лучше купите эту чертову селедку в ближайшем магазе и забудете про кулинарию надолго. Если же вы женщина, то муж вам после такого угощения пожизненно обязан. Цените своё время!!!

Мне всегда больно смотреть на людей, которые тратят часы на приготовление блюда, которое того не стоит, просто не стоит. Подумайте: блинчики, пирожки и булочки, сложные салаты, фритюр, бурфи, жареный сыр, даже плов, жюльен и халава — это всё бесполезная трата времени, вред для здоровья и издевательство над едой. Ведь в итоге, как бы вы не старались, это окажется внутри, а балласт выйдет известным путём. Не становитесь заложниками времени и жертвами блюдомании. Оцените трезво, что действительно нужно вашему телу, а что навязано извне.

Поэтому\ldots

Я подобрал для вас полезные, простые и функциональные блюда, из которых можно составить полноценное меню на неделю. Описание сокращено до минимума, чтобы рецепт можно было за 2 секунды окинуть взглядом и понять принцип процесса.

\section{Про самообразование}

А теперь напишу о том, как сделать всё ещё проще и в радость. Некоторые советы для начинающих, чтобы популярные вопросы отпали сразу:
\begin{itemize}
\item  Купите хорошую доску, набор удобных для вас ножей, посуды и других инструментов, чтобы приготовление еды приносило радость, и не превращалось в каторгу. Вам скорее всего понадобятся: небольшой казан, посуда нержавейка с толстым дном 2х-3х объёмов, лопатка, глубокое сито с ушками + миска под него, половник, большая ложка, большой широкий нож, малый нож, картофелечистка, тёрка, электрический чайник, стационарный и погружной блендер с мельничкой, кофемолка, духовка, плита и силиконовый противень.
\item  Установите обратноосмотический фильтр, забудьте про накипь в чайнике и наслаждайтесь чистой водой.
\item  Помните, что газовая плита всегда лучше конвекторной, а живой огонь лучше газа.
\item  Поинтересуйтесь в интернете о том, как чистить привычные вам продукты и вы узнаете много нового. Хороший пример: арбуз, ананас, авокадо или гранат. Возможно я даже сделаю подборку видео на тему.
\item  Научитесь эффективно и быстро резать. Это сэкономит вам тысячи часов вашей жизни.
\item  Режьте продукты так, чтобы: они сохраняли вкус, помещались в ложку и быстро готовились.
\item  Если падает нож, не спешите его ловить.
\item  Храните крупы и специи, свежевыжатые соки, супы и приготовленные блюда герметично в контейнерах.
\item  Узнайте вообще, как и где лучше хранить разные продукты.
\item  Узнайте время варки продуктов.
\item  Пробуйте и нюхайте каждый продукт перед закладкой. Помните о сроках годности.
\item  Мойте крупу перед варкой.
\item  В сладкие блюда добавляйте щепотку соли, а в солёные немного сахара.
\item  Помните, что мука — это клей, а сахар — яд.
\item  Всегда ищите лучшие пути для выполнения привычных операций.
\end{itemize}

\begin{quote}
    \emph{``Для многих кулинария — это творчество. Но в мире творчества есть тысячи гораздо более интересных занятий для души. Помните, что это всего лишь пища и её главная задача — питать наше тело.''}

\end{quote}


\section{Универсальный язык хорошего вкуса}

Итак, сегодня раскрываю секреты\ldots которые секретами не являются.
Как же сделать пресное блюдо привлекательным? 
Очень просто: наделить его всеми шестью вкусами — солёным, сладким, кислым, горьким, вяжущим и острым. На этом всё, спасибо за внимание, до свидания\ldots Шучу, читаем дальше!

Как известно, в ведической традиции считается, что основная трапеза должна содержать все 6 вкусов. Я с этим утверждением абсолютно солидарен, хоть и не приветствую соль. Когда эти вкусы гармонично взаимодействуют, человек получает весь спектр эмоций, потому что по сути, вкус~--- это эквивалент эмоции, и как только вы поймёте это, вам станет ясно, почему иногда хочется шоколадки, а иногда и хрена с чесноком. В другом случае, это потребность физиологического характера, например в период очищения или дефицита определённых веществ.

Так вот. Есть блюда, любимые многими, такие как плов, например. Я здесь не рассматриваю мясо, потому что с ним можно хоть гвозди есть, всё равно будет вкусно. Речь о вегетарианском варианте. Чтобы вы поняли, что делает безвкусный рис таким вкусным, надо просто проанализировать состав плова:
\begin{itemize}
\item Соль — солёный
\item Морковь, изюм, курага и чернослив — сладкий
\item Барбарис — кислый
\item Куркума, шафран — горький
\item Кумин — вяжущий
\item Лук и чеснок — острый
\item ну и Рис между делом, как сорбент всей композиции
\end{itemize}

Я думаю, вы теперь поняли, что рис без специй будет напоминать что-то вроде риса из студенческой столовки. Плов не самый лучший пример, но показательный. Я не считаю рис достойным продуктом для употребления, но на востоке в этом очень преуспели.

Теперь вы можете понять почему базу составляют всего несколько специй.

Другой пример — салат с рукколой и черри. Здесь всё дело в соусе по тому же принципу:
\begin{itemize}
\item оливковое масло, 2 стл
\item лимонный сок, 2 стл — кислый
\item гранатовый соус или шиповника, 1 стл — сладкий
\item чёрный перец, щепотка — острый
\item прованские травы, 1 чл — вяжущий
 
\item руккола — горький
\item сушёные помидоры — солёный
\end{itemize}


Заправка для Греческого салата:
\begin{itemize}
\item 3 чл оливкового масла
\item 3 чл лимонного сока
\item 1 зуб чеснока
\item 1/4 чл орегано
\item 1/4 чл соли
\item Сладость даёт болгарский перец
\end{itemize}

Ещё один вариант соуса:
\begin{itemize}
\item Растительное масло, 2 стл
\item Чеснок, 4 зуб.
\item Соевый соус, 4 стл
\item Лимонный сок, 2 чл
\item Мёд, 1 чл
\item Черный перец, 2 щеп.
\item Зелень, горсть.
\end{itemize}

Как вы видите, соусы содержат в себе почти все вкусы, а недостающие восполняют овощи и фрукты. Вообщем, принцип понятен.

Я заметил, что добавляя в блюдо чеснок, оно буквально преображается. Можно долго рассуждать о его раджастической природе, что это продукт в гуне невежества и прочее-прочее, но мы живём на земле, в конце концов. Особо просветлённые могут заменить его асафетидой. Чеснок, горький и острый одновременно, делает блюдо аппетитнее, это факт. В ресторанах вместе с лавашом часто подают пиалу с чесночным маслом. А на столе всегда есть соль, сахар и перец и лимон.

Специи~--- самый доступный способ сделать блюдо богаче. Только не сублимируйте эмоциональный вакуум — любимое дело гораздо важнее еды! Любите себя и дарите любовь.


\begin{figure}[ht]
    \centering
    \includegraphics[width=0.6\textwidth]{img/SixTastes}
    \label{fig:1}
\end{figure} 

\section{Про специи}
Сегодня я хотел бы поговорить о специях и о том, как их выбирать и хранить. Об их свойствах я рассказывать не буду, потому что это долго, неинтересно и больше про лечебные свойства, что в наш динамичный мир не вписывается. Поскольку большинство людей — заложники вкуса, то лишь его и обозначу. Ниже перечислю основные специи, как говорится must-have, которые должны быть всегда под рукой.

БАЗА:

Кумин (зира) — вяжет: в основном, для бобовых
Кориандр (семена кинзы) — для овощей и нотки «бородинского»
Корица — делает всё слаще
Чили — жгучая острота
Чёрный перец — грубая острота
Ванильный сахар — делает выпечку сексуальнее
Соль — делает так, что хочется ещё
Сахар — без комментариев
Мёд — сладость без вреда
Лимон — добавляет кислинку
Чеснок — делает всё аппетитнее, остро-горький
Зелень (кинза, петрушка, укроп, руккола) — вкус, аромат и полезная горечь

НА РЕДКИЙ СЛУЧАЙ:

Куркума — для овощей
Гвоздика — для выпечки
Мускат — универсален
Имбирь — острый, универсален
Бадьян — пряный, для напитков
Хмели-сунели — для каш
Розовый перец — для супа
Прованские травы — для салатов
Орегано — для помидоров
Лавровый лист — для супа

ЦЕЛЫЕ ИЛИ МОЛОТЫЕ?

Целые предпочтительнее, т. к. дольше сохраняют аромат и вкус. Поэтому лучше иметь кофемолку и молоть самостоятельно. Однако есть специи, которые смолоть в обычной кофемолке не получится, поэтому надо брать готовый вариант. Обычно это Корица, Мускат, Куркума и другие твёрдые виды.

КАК ВЫБИРАТЬ И ГДЕ ПОКУПАТЬ:

Лучшие специи, на мой скромный взгляд, производит компания TRS. Приобрести их вы можете здесь: http://indianspices.ru или любом другом магазине специй. Если нет TRS, берите любые в герметичной упаковке. Всё, что продаётся на рынках в открытом виде — давно выветрилось и интереса не представляет.

КАК ХРАНИТЬ:

Самое простое — в аптечных контейнерах (видел также в Ашане по 15р/шт). Купите по одному большому для каждой отдельной специи. Это дёшево и практично. Всё остальное — маркетинг. В таком виде они хранятся указанный на упаковке срок.
Если блюдо содержит смесь специй, как плов или пряники, то композицию лучше готовить непосредственно перед закладкой.

ССЫЛКИ:

специи:
http://indianspices.ru 
http://ashaindia.ru/indiyskie-specii/
гималайская соль:
http://hpcsalt.ru

-------
Здесь я просто делюсь своим опытом. Большинство людей блуждают в потёмках невежества и до сих пор употребляют плоть животных. Надеюсь, что публикуемые рецепты вдохновят вас подняться на ступень выше, и взглянуть на свой рацион трезво.

Завтра я расскажу вам маленький секрет хорошего соуса, овладев которым, вы сможете сделать любое блюдо аппетитным… и объясню, почему в базу вошли указанные специи.

