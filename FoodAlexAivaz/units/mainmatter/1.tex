\chapter{Советы \& лайфхаки}

\section{Почему все так просто}

В моей подборке нет таких блюд, как щи или картошка, потому что я считаю их абсолютно пустыми, также нету блюд, которые готовятся по несколько часов, в несколько этапов, во фритюре и т.д. Я призываю вас уходить от подобной кулинарной глупости.

Да, я могу испечь печенье для мамы, и только потому, что она просит и я её очень люблю. Но это отнимает уйму времени и грязной посуды. Если вы мужчина и решили приготовить какие-нибудь пирожки, селедку под шубой или фалафель, то скорее всего либо выдохнетесь, либо вообще разочаруетесь и лучше купите эту чертову селедку в ближайшем магазе и забудете про кулинарию надолго. Если же вы женщина, то муж вам после такого угощения пожизненно обязан. Цените своё время!!!

Мне всегда больно смотреть на людей, которые тратят часы на приготовление блюда, которое того не стоит, просто не стоит. Подумайте: блинчики, пирожки и булочки, сложные салаты, фритюр, бурфи, жареный сыр, даже плов, жюльен и халава — это всё бесполезная трата времени, вред для здоровья и издевательство над едой. Ведь в итоге, как бы вы не старались, это окажется внутри, а балласт выйдет известным путём. Не становитесь заложниками времени и жертвами блюдомании. Оцените трезво, что действительно нужно вашему телу, а что навязано извне.

Поэтому\ldots

Я подобрал для вас полезные, простые и функциональные блюда, из которых можно составить полноценное меню на неделю. Описание сокращено до минимума, чтобы рецепт можно было за 2 секунды окинуть взглядом и понять принцип процесса.

\section{Про самообразование}

А теперь напишу о том, как сделать всё ещё проще и в радость. Некоторые советы для начинающих, чтобы популярные вопросы отпали сразу:
\begin{itemize}
\item  Купите хорошую доску, набор удобных для вас ножей, посуды и других инструментов, чтобы приготовление еды приносило радость, и не превращалось в каторгу. Вам скорее всего понадобятся: небольшой казан, посуда нержавейка с толстым дном 2х-3х объёмов, лопатка, глубокое сито с ушками + миска под него, половник, большая ложка, большой широкий нож, малый нож, картофелечистка, тёрка, электрический чайник, стационарный и погружной блендер с мельничкой, кофемолка, духовка, плита и силиконовый противень.
\item  Установите обратноосмотический фильтр, забудьте про накипь в чайнике и наслаждайтесь чистой водой.
\item  Помните, что газовая плита всегда лучше конвекторной, а живой огонь лучше газа.
\item  Поинтересуйтесь в интернете о том, как чистить привычные вам продукты и вы узнаете много нового. Хороший пример: арбуз, ананас, авокадо или гранат. Возможно я даже сделаю подборку видео на тему.
\item  Научитесь эффективно и быстро резать. Это сэкономит вам тысячи часов вашей жизни.
\item  Режьте продукты так, чтобы: они сохраняли вкус, помещались в ложку и быстро готовились.
\item  Если падает нож, не спешите его ловить.
\item  Храните крупы и специи, свежевыжатые соки, супы и приготовленные блюда герметично в контейнерах.
\item  Узнайте вообще, как и где лучше хранить разные продукты.
\item  Узнайте время варки продуктов.
\item  Пробуйте и нюхайте каждый продукт перед закладкой. Помните о сроках годности.
\item  Мойте крупу перед варкой.
\item  В сладкие блюда добавляйте щепотку соли, а в солёные немного сахара.
\item  Помните, что мука — это клей, а сахар — яд.
\item  Всегда ищите лучшие пути для выполнения привычных операций.
\end{itemize}

\begin{quote}
    \emph{``Для многих кулинария — это творчество. Но в мире творчества есть тысячи гораздо более интересных занятий для души. Помните, что это всего лишь пища и её главная задача — питать наше тело.''}

\end{quote}

\section{Про растительные масла}

Про вред рафинированных исписан весь интернет, читайте сами. Здесь про нерафинированные холодного отжима

Если рассматривать растительные масла как функциональный продукт, то по действию они все примерно одинаковы — помогают выводить жирорастворимую слизь со стенок кишечника, и различаются только ароматом. 

Для заправки овощных салатов (под овощами я имею ввиду корнеплоды и зелёную массу) можно брать самое простое подсолнечное холодного отжима. Нормальное масло отжимается на деревянном прессе, не нагревается и практически не пахнет, в отличие от дешёвых нерафинированных. Прийдётся поискать, но оно того стоит.

А для удовольствия можете брать любое по душе и кошельку: кедровое, кунжутное, грецкого ореха, тыквенное — просто для аромата.
Не спорю, есть и лечебные масла, как облепиховое, черного тмина, и прочие, но их пользу я не рассматриваю, потому что считаю костылями, которые при неадекватном питании не помогут.

Раз уж разговор зашёл об очищении, вот вам салатик для обнуления кармы:
\begin{itemize}

\item морковь и китайская капуста 500
\item кориандр 1 чл
\item чили 2-4 щепотки
\item мёд 2 чл
\item лимонный сок 20 мл
\item зелень 2 горсти
\item масло растительное х/о 2-4 стл
\end{itemize}

Корнеплоды на крупной тёрке и заправить специями. Хорошая метла для тех, кто давно не чистился. 


Здравия!


\section{Про голодание}
Экадаши — ведический пост и лучший день для разгрузки, как умственной, так и физической. Хочу поделиться с вами методиками «Шанкх-пракшалана» — очистки ЖКТ солёной водой.

Я сам из тех людей, для которых очищение началось c физического плана, и пракшалана — то, что я практикую и по сегодняшний день. На живом питании чистка проходит вообще легко и быстро, чего и вам желаю. Проведя уже несколько десятков процедур, имею право поделиться своим опытом с вами. Если вы не уверены, что стоит чиститься, то скорее всего вы либо ещё не готовы, либо уже питаетесь соками и дальше можете не читать. Для остальных прикладываю статьи, написанные с любовью и претерпевшие несколько редакций.
\begin{itemize}
    \item \hyperref[fasting1]{Пракшалана кратко и с нюансами}
    \item \hyperref[fasting2]{Клизма по фэншуй}
    \item \hyperref[fasting3]{Про экадаши}
\end{itemize}

Это не волшебная таблетка, но верный шаг на пути к здоровью. Чистое тело — ясный ум. Включайте мозг, и «пусть пища станет вашим лекарством». 

Информации по ссылкам предостаточно. 
Всем здравия и лёгкости в теле!

\subsection{Пракшалана кратко и с нюансами}\label{fasting1}
Здесь я хочу озвучить некоторые детали касательно практики, которые помогут вам существенно облегчить процедуру.
Суть практики заключается в очищении всего жкт с помощью солёной воды. Она не всасывается кишечником и потому полностью выходит через задний проход. Цель всей процедуры — добиться того, чтобы получить на выходе максимально чистую воду без примесей.

Процедура по канонам начинается утром с 5 до 7 (но на практике, нормально проходит вплоть до полудня) и занимает обычно 1–2 часа (в особо запущенных случаях и дольше). Этот день надо освободить от всех дел, и желательно совместить с днём голода (удобнее всего на Экадаши). За день до процедуры необходимо оставить в рационе только овощи и фрукты, а лучше пюрированную пищу или соки. В 18 часов приём твёрдой пищи прекращается! И это самый важный из всех пунктов, пренебрегая которым, вся утренняя процедура может превратиться в адские муки. Ложиться пораньше, с утра ничего не пить.
С вечера необходимо подготовить от 3 до 6 литров чистой воды и поваренную соль без добавок. Идеально, если вода будет после осмоса или дистиллят — солёную воду без посторонних примесей пить легче. Лечь спать надо пораньше, чтобы с утра иметь в запасе достаточно энергии для процедуры — кишечник должен быть в тонусе!

С утра сделать одну-две 3л клизмы (в кружке Эсмарха) с 1 ст.л. соли и 1 ч.л. соды — это поможет провести утреннюю процедуру намного! быстрее. Далее следует контрастный душ. Также необходимо обеспечить приток свежего воздуха. Подсолите воду в кастрюле с расчётом 15 г соли на 1 литр воды (не больше и не меньше), и нагрейте на плите до температуры тела (воду периодически надо будет подогревать). Может понадобиться до 6 литров. Брезгливые чистюли могут вместо кружки Эсмарха использовать гигиенический душ.
Постойте в берёзке 2 минуты для того, чтобы открыть привратник желудка, и можно приступать.

\textbf{Суть}

Энергично выпиваете стакан солёной воды и делаете \href{http://youtu.be/-jrGWPXai2g}{серию упражнений}. В каждом упражнении 4 поворота в каждую сторону, начиная с правой. Затем ещё стакан, опять серию упражнений и т.д. — в нормальном темпе. В определённый момент захочется в туалет (время у каждого своё).

\textbf{Детали}
\begin{itemize}

\item С утра вы должны быть в тонусе! Если чувствуете упадок сил — моторика кишечника тоже будет вялой и продуктивность упадёт в разы (в таком случае процедуру лучше перенести на другой день).
\item Для клизмы используйте вазелин и смажьте обильно задний проход снаружи и изнутри — это облегчит процедуру и защитит нежную слизистую от раздражения после соли. Насадка для клизмы не нужна — трубку обрезать и аккуратно обжечь — так вода будет литься быстрее. Если вы не делаете клизму, в любом случае используйте вазелин, чтобы задний проход не горел огнём от соли.
\item Одевайтесь легко. Как только захотелось в туалет, идите незамедлительно. Желание опорожниться может случиться во время выполнения упражнений или питья воды, — это нормально: идите освободите кишечник и возвращайтесь. Засиживаться не надо, ждать эвакуации тоже. Туалетной бумагой надо пользоваться очень осторожно, не травмируя слизистую. Всё выполняется оперативно, но без спешки.
\item Если вы устали или вода где-то застряла, полежите пару минут, кишечнику надо иногда отдыхать. Затем сделайте несколько серий упражнений.
\item Если желудок переполнен, давиться новой порцией не надо — сделайте несколько серий упражнений, и если вода не пошла вниз, повторите несколько раз первое упражнение, или постойте ещё раз в берёзке 2 минуты.
\item Если после 8-го стакана вода не выходит, значит либо вчера вы плотно поели, либо не выспались, либо закрыты какие-то клапаны кишечника — в последнем случае можно сделать небольшую клизму.
\item Выпить надо столько, чтобы на выходе получить чистую воду без примесей. Возможен жёлтый оттенок, ввиду того, что желчь из печени также мобилизуется.
\item Если заболела голова, значит соли внутри уже слишком много и процедуру надо заканчивать и начинать пить чистую воду. Не самое приятное, но такое бывает.
\item Если вам трудно встать в 5 утра, — ничего страшного. Практика показала, что и в полдень процедура проходит успешно.

\end{itemize}

После этого традиционно рекомендуется промыть желудок чистой тёплой водой для того, чтобы закрыть привратник желудка. 5–6 стаканов до рвоты, наклонившись вниз с головой ниже пояса. Можно обойтись и без этой процедуры, клапан закроется сам, но несколько стаканов чистой воды выпить надо, чтобы удалить остатки соли из желудка. Чистая вода покажется вам очень желанной, если раньше вы пили её недостаточно.

На этом очистка закончена, можно выпить воды и часок полежать. В этот день было бы замечательно поголодать 24 часа на дистилляте и дать ЖКТ долгожданный отдых.
Вегетарианский классический выход осуществляется через варёный рис и топлёное масло (хотя я это категорически не приветствую, — лучше используйте гречку или тушёные овощи). Для сыроедов: если голодаете, то выход через корнеплоды на следующее утро; если голодать не планируете, то первым приёмом пищи могут быть любые разбавленные соки.

На протяжении следующих 2–6 часов, в зависимости от вашей физической активности, из вас будут выходить остатки воды. Имейте это ввиду, если планируете покинуть дом.

Практикуя раз в две недели такую чистку следует добиться результата прохождения воды сверху вниз за 20–30 минут (примерно 6й–8й стакан), а всей процедуры за 1 час (зависит от вашей текущей физической активности и питания).

\textbf{Противопоказания}

Шанк-Пракшалана противопоказана при язве желудка в стадии обострения, и других острых заболеваниях органов пищеварения. То же самое относится к лицам, страдающим острым поражением пищеварительного тракта: дизентерией, поносом, острым колитом, гастрите, панкреатите острым аппендицитом, туберкулезом и раком кишок. Однако описываются и случаи излечения этих заболеваний Шанк-Пракшаланой.
Процедура не рекомендована  в период менструаций, при высокой температуре или давлении,  запрещена при сердечных приступах. Также она запрещена во время беременности. После трехдневного голодания её можно делать не ранее чем через неделю после полного выхода, после недельного голодания – не ранее чем через месяц.

\textbf{Данная рекомендация основана лишь на моём личном многолетнем опыте. За всё, что вы делаете со своим организмом — несёте ответственность только вы.}


\subsection{Клизма}\label{fasting2}
Итак, пару слов о самом приятном \faSmileO.

Клизма в данном случае — это кружка Эсмарха на 2 литра (а груши, к сведению, называются спринцовками).
В аптеке нужно купить её и тюбик вазелина, в продуктовом — поваренную чистую соль и соду. + 3л стеклянную банку.
\\

Насадки не понадобятся. Нужно обрезать и обжечь конец трубки на огне, чтобы не травмировать слизистую. Так вода будет поступать быстрее. Клизму и трубку необходимо обдать кипятком изнутри и снаружи после покупки для стерилизации. 
\\

Сделать петельку на кружке и подвесить на крючок в туалете. Если крючка нет, можно на рукоятку швабры — нижний край кружки должен быть не ниже 75 см от пола.

Смазать вазелином конец трубки и анальное отверстие изнутри и снаружи — это защитит нежную слизистую от раздражения. 

Подготовить раствор: на 3л тёплой воды $37-39^{\circ}$ С (Т.е. чуть теплее, чем подмышкой. Если холодней — то будут спазмы, если горячей — ожоги) — 1 стл соли и 1 чл соды с горками.

Штатный фиксатор на трубке нам тоже не понадобится, её надо поднять выше края кружки, влить раствор в кружку, пережать z-образно конец трубки.

Встать в коленно-локтевую позицию, прогнуть поясницу, вставить конец трубки в анальное отверстие, разжать трубку — вода начнёт поступать. Если сразу давит, значит надо опустошиться и делать дальше — редко, когда с первого подхода можно влить 1,5 литра. Внутри кишечника много поворотов и препятствий — можно помогать себе мышцами живота, проталкивая воду дальше. Не надо надрываться, но постарайтесь почувствовать, где именно сейчас вода. До боли доводить не надо. Вот и всё — ничего хитрого.
\\

Компетентные люди частить с клизмами не рекомендуют, и я придерживаюсь такого же мнения. В любом случае, — по ощущениям.
\\

\textbf{Данная рекомендация основана лишь на моём личном опыте. За всё, что вы делаете со своим организмом — несёте ответственность только вы.}


\subsection{Учет лунного цикла при проведении голодания}\label{fasting3}

Лунный цикл накладывает наиболее сильный отпечаток на процессы, происходящие в организме человека. Он самый важный, ибо в нем имеются дни и целые периоды, когда организм сам очищается, и дни , когда это делать нежелательно.

Движение луны, ее фазы вызывают на Земле приливы и отливы. Отражение этого процесса наблюдается и в человеческом организме в виде двух явлений.
\begin{enumerate}
\item Наш организм состоит из воды и поэтому следует за приливами и отливами.
\item От изменяющегося гравитационного воздействия со стороны Луны наш организм становится то легче, то тяжелее. Когда он становится легче, он "расширятся", что благоприятствует очистительному процессу голодания; когда он становится тяжелее, он сжимается под действием гравитации Земли и собственных сил. Ткани "зажаты" и отдают шлаки с большим трудом.
\end{enumerate}
Если действовать в согласии с лунными циклами~--- вам обеспечен успех.

\textbf{Как использовать лунный цикл?}
\begin{enumerate}
\item Голодание до 7 суток проводите только во II и IV фазы Луны.В это время организм естественно очищается и вы этому способствуете.

\item Голодание больше 7 суток приурочивайте так, чтобы большинство дней голода приходилось на II и IV фазы.
\item  Голодание больше 14 дней планируйте так, чтобы выход из голода совпал с началом лунного цикла. В это время организм естественно запускает жизненные процессы и вы без всяких затруднений войдете в ритм его работы. Если вы уверены в себе и имеете опыт длительных сроков голодания, то начинайте голодать в начале лунного цикла, а выходить после его завершения. Это будет наилучший вариант.
\end{enumerate}

Однодневные, полуторасуточные голодания лучше всего проводить в дни экадаши: 11-й день после новолуния и 11-й день - после полнолуния. Древние мудрецы подметили, что в эти дни Земля становится "влажной", то есть лунная гравитация поднимает воду из глубины ближе к поверхности. Подобное происходит и в нашем теле, что способствует гораздо лучшему очищению, нежели в другие дни. Кроме того, эти дни считаются энергетически сильными, и вы легко перенесете голод и прекрасно очиститесь.

\textbf{Когда лучше выполнять клизмы при голоде}

В сутках имеются два периода, во время которых особенно активны очистительные органы. Утром, когда конденсируется воздух и выпадает роса, в человеческом организме также наблюдается подобное явление. С 5 до 7 утра активен толстый кишечник и происходит удаление отходов пищеварительного процесса. Вечером с 17 до 19 часов наступает период затишья, в котором активна работа почек по выведению из организма продуктов белкового обмена. Поэтому лучше всего очистительные клизмы проводить именно в это время.

Из всего вышенаписанного вытекает правило: \textbf{голодать нужно в соответствии с природными процессами}, а не с произвольно выбранными датами. Например,голодающие раз в неделю в определенные дни периодически "наскакивают" на неблагоприятные дни лунного месяца и вредят своему здоровью.
Подавляющему большинству людей вообще рекомендуется воздерживаться от пищи только в дни экадаши, и это будет оптимально по всем параметрам. Все остальные рекомендации основаны на чем угодно, но только не на Законах Природы.

(\href{https://vk.com/away.php?to=http%3A%2F%2Fgovoritluna.ru%2Flechebnoe-golodanie%2Fuchet-lunnogo-tsikla-pri-provedenii-golodaniya&post=2453299_2575&cc_key=}{источник})

