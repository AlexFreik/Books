\chapter*{Соусы}
\label{sec:sauces}
\addcontentsline{toc}{section}{\nameref{sec:sauces}}


%%% ================== recipe ================== %%%
\recipe{Айолли}{5}{0.35 кг}
{\label{aiolli}
\item Семечки подсолнуха сырые 300 мл
\item Кинза – 2 горсти
\item Соевый соус – 4 стл
\item Сок лимона – 4 стл
}{
\item Чеснок – 2 зуб
}{
Семечки измельчить в кофемолке, остальное в блендере, затем смешать. С хлебцами отлично!
}{
\begin{advice}
\item Берите соус Sensoy естественного брожения в стеклянной бутылке.
\item Паштет из семечек, нравится всем!
\end{advice}}{}



%%% ================== recipe ================== %%%
\recipe{Тахини}{5}{0.3 кг}
{\label{takini}
\item 1 стакан белого кунжута 
\item 1 стл кунжутного масла
\item 3 стл лимона
\item 2 стл петрушки
}{
\item 2 зуб чеснока
\item щепотка чили
\item 1/2 чл соли
\item 1/2 чл кумина (по желанию)
}{
Кунжут в кофемолке, остальное — в мельничке. Смешать. 
}{
\begin{advice}
\item Получается плотно, по вкусу прямо как сыр — можно на хлебцы намазывать.
\end{advice}}{}



%%% ================== recipe ================== %%%
\recipe{Ткемали из желтой сливы}{5}{250 г}
{
\item Слива жёлтая 300 г
}{
\item Семя кориандра, 1 чл
\item Семя укропа, 1 чл
\item Чеснок 1 небольшой зуб
\item чили, 1/4 чл
\item Соль, 1 чл
}{
Всё в блендер.  
}{
\begin{advice}
\item Очень вкусный соус! 
\end{advice}}{}



%%% ================== recipe ================== %%%
\recipe{Хренодёрка (хреновина)}{}{1 кг}
{
\item 1 кг вкусных помидоров
\item 40 г хрена
\item 3 стл лимонного сока
\item 2 чл мёда
}{
\item 50 г чеснока
\item 2 чл соли
\item 1/2 чл чили 
}{
Всё перетереть в мясорубке с крупной решёткой, закатать в стерильные банки.
}{}{khren}



%%% ================== recipe ================== %%%
\recipe{ХУМУС}{}{}
{
\item 1 ст отваренного нута или фасоли
\item 10 мл лимонного сока
\item 60 мл воды для нута, или 30мл для фасоли
\item 1 чл кунжутного масла
\item несколько веток кинзы 
}{
\item 1/2 чл соли
\item 1/4 чл зиры
\item 1/4 чл кориандра
\item 1 щепотка чили
\item 1 щепота черного перца
\item 1 зуб чеснока
}{
Всё смешать в мельничке.
}{
\begin{advice}
\item Если лень варить бобы~--- возьмите готовые в банке. 
\item Специи молотые. 
\item Рецепт выверен, смело готовьте. Классный паштет и самостоятельное блюдо.
\item Хранится 3 дня в холоде.
\item Не рекомендую экспериментировать с нутовой мукой. Схалтурить не получится.
\end{advice}}{humus}



