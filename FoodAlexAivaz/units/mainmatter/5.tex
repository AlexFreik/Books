\chapter*{Салаты}
\label{sec:salats}
\addcontentsline{toc}{section}{\nameref{sec:salats}}

Почему мало салатов?

Действительно, какая несправедливость!
Может быть, единственный салат, который вам пригодится в жизни~--- \hyperref[cleanSalat]{салат для очищения}.


Почему нет салатов из огурцов и помидоров и прочей дребедени? Потому что в отличие от приготовленных, свежие плоды употребляются отдельно, а в противном случае вызывают несварение, брожение и вздутие. Поэтому я исключил подобные кулинарные извращения во благо здоровья.

По этой же причине не найдёте здесь всеми любимой шарлотки и грибного супа, как изделий, вызывающих изжогу и запоры. По этой же причине я не пропагандирую хлеб, даже бездрожжевой.

По-хорошему, мне следовало бы исключить и кое-где фигурирующие молочные продукты, но постарайтесь почувствовать самостоятельно, что конкретно вам полезно, а что вызывает болезни. И прежде чем что-то исключать, найдите достойную альтернативу.
Об альтернативах можете \hyperref[sec:replace]{прочитать выше}.

Успехов!




%%% ================== recipe ================== %%%
\recipe{Салат для очищения}{}{}
{
\item морковь и китайская капуста поровну 250
\item зелень горсть
\item масло растительное х/о 1-2 стл
}{
\item кориандр 1/2 чл
\item чили 2 щепотки
\item мёд 1 чл
\item лимонный сок 10 мл
}{
Морковь на крупной тёрке, капусту и зелень порезать грубо, и заправить специями. Умножайте в несколько раз на большую кастрюлю.
}{
\begin{advice}
\item Почему морковь и китайская капуста? Потому что нейтрально, дёшево, эффективно и без побочных эффектов. 
\item Ничего другого, типа перца или помидоров, пожалуйста не добавляйте, т.к. всё равно не усвоится. 
\item Если вам скучно, поэкспериментируйте с другими корнеплодами и зеленью. Однако я уже третий год делаю один и тот же салат, и мне ещё не не наскучил.

\item Можно практически не жевать, потому что как вошёл, так и выйдет, но захватит с собой всё то, что в кишечнике быть не должно. Это грубая клетчатка – хорошая метла для тех, кто давно не чистился. Также можно есть в любое время суток, потому что ферменты здесь уже не участвуют.
\item Если у вас стул реже 3х-4х раз в день, — этот салат для вас. 
Гастроэнтерологи с противоположной точкой зрения могут идти мимо, не переубедите.


\item Через три недели ежедневного употребления салата оцените результаты. 

\end{advice}
}{cleanSalat}




%%% ================== recipe ================== %%%
\recipe{Морковь по-корейски}{5}{0.3 кг (1 порция)}
{
\item сладкая морковь 250г
\item масло подсолнечное х/о 1 стл (по желанию)
\item сок лимона 1,5 стл
\item мёд 1 чл
\item кориандр молотый 1/2 чл,
\item кинза, горсть
}{
\item кунжут 1 стл,
\item чеснока зубчик 1 шт
\item перец черный 1 щеп,
\item перец чили 1 щеп,
\item соль 1/2 чл
}{
Морковь и чеснок потереть. Всё смешать и оставить в закрытой таре в холоде на 3-8 часов. Умножайте в несколько раз — съедается мгновенно!
}{}{carrot}



%%% ================== recipe ================== %%%
\recipe{Салат с рукколой}{5}{0.5кг (2 порции)}
{
\item руккола 50
\item помидоры черри 400
\item сушёные помидоры (по желанию)
\item кедровые орехи, горсть
}{
\item оливковое масло, 2 стл (по желанию)
\item лимонный сок, 2 стл
\item гранатовый соус или шиповника, 1 стл
\item чёрный перец, щепотка
\item прованские травы, 1 чл
}{
Сделайте соус из <<Специй>> и добавьте к порезанным овощам.
}{
\begin{advice}
\item Да, можете туда кинуть какую-нибудь моцареллу или сулугуни.
\item Если нет кедровых~--- добавьте кешью.

\end{advice}}{salatTomato}


