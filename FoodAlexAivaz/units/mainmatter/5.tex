\newpage
\section{Салаты}
\label{sec:salats}
%\addcontentsline{toc}{section}{\nameref{sec:salats}}

Почему мало салатов?

Действительно, какая несправедливость!
Может быть, единственный салат, который вам пригодится в жизни~--- \hyperref[cleanSalad]{салат для очищения}.


Почему нет салатов из огурцов и помидоров и прочей дребедени? Потому что в отличие от приготовленных, свежие плоды употребляются отдельно, а в противном случае вызывают несварение, брожение и вздутие. Поэтому я исключил подобные кулинарные извращения во благо здоровья.

По этой же причине не найдёте здесь всеми любимой шарлотки и грибного супа, как изделий, вызывающих изжогу и запоры. По этой же причине я не пропагандирую хлеб, даже бездрожжевой.

По-хорошему, мне следовало бы исключить и кое-где фигурирующие молочные продукты, но постарайтесь почувствовать самостоятельно, что конкретно вам полезно, а что вызывает болезни. И прежде чем что-то исключать, найдите достойную альтернативу.
Об альтернативах можете \hyperref[sec:replace]{прочитать выше}.

Успехов!




%%% ================== recipe ================== %%%
\recipe{Салат для очищения}{5}{0.25}
{\label{cleanSalad}
\item 250 г моркови и китайской капусты поровну
\item 1-2 стл растительного масла х/о 
\item 1 горсть зелени
}{
\item 0.5 чл кориандра
\item 2 щеп чили
\item 1 чл мёда
\item 10 мл лимонного сока
}{
Морковь на крупной тёрке, капусту и зелень порезать грубо, и заправить специями. Умножайте в несколько раз на большую кастрюлю.
}{
\begin{advice}
\item Почему морковь и китайская капуста? Потому что нейтрально, дёшево, эффективно и без побочных эффектов. 
\item Ничего другого, типа перца или помидоров, пожалуйста не добавляйте, т.к. всё равно не усвоится. 
\item Если вам скучно, поэкспериментируйте с другими корнеплодами и зеленью. Однако я уже третий год делаю один и тот же салат, и мне ещё не не наскучил.

\item Можно практически не жевать, потому что как вошёл, так и выйдет, но захватит с собой всё то, что в кишечнике быть не должно. Это грубая клетчатка – хорошая метла для тех, кто давно не чистился. Также можно есть в любое время суток, потому что ферменты здесь уже не участвуют.
\item Если у вас стул реже 3х-4х раз в день, — этот салат для вас. 
Гастроэнтерологи с противоположной точкой зрения могут идти мимо, не переубедите.

\item Через три недели ежедневного употребления салата оцените результаты. 

\end{advice}
}{cleanSalatCopy}  % change photo




%%% ================== recipe ================== %%%
\recipe{Морковь по-корейски}{5}{0.3}
{
\item 250 г сладкой моркови
\item 1 стл подсолнечного масла х/о (по желанию)
\item 1.5 стл лимонного сока 
\item 1 чл мёда
\item 1 стл кунжута
\item 1 горсть кинзы
}{
\item 0.5 чл кориандра молотого
\item 1 зубчик чеснока
\item 1 щеп черного перца
\item 1 щеп перца чили
\item 0.5 чл соли
}{
Морковь и чеснок потереть. Всё смешать и оставить в закрытой таре в холоде на 3-8 часов. Умножайте в несколько раз — съедается мгновенно!
}{}{carrot}  % change photo



%%% ================== recipe ================== %%%
\recipe[Соус]{Салат с рукколой и черри}{5}{0.5}
{\label{arugula}
\item 50 г рукколы
\item 400 г помидоров черри
\item Сушёные помидоры (по желанию)
\item 1 горсть кедровых орехов
}{
\item 2 стл оливкогого масла (по желанию)
\item 2 стл лимонного сока
\item 1 стл гранатового соуса или соуса шиповника
\item 1 чл прованских трав
\item 1 щеп чёрного перца
}{
Сделайте соус и добавьте к порезанным овощам.
}{
\begin{advice}
\item Да, можете туда кинуть какую-нибудь моцареллу или сулугуни.
\item Если нет кедровых~--- добавьте кешью.
\end{advice}}{salatTomato}


