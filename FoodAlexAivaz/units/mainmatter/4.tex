\chapter*{Гарниры}
\label{sec:garnish}
\addcontentsline{toc}{section}{\nameref{sec:garnish}}



%%% ================== recipe ================== %%%
\recipe{Лобио}{}{1.3}
{
\item 1 ст сухой  фасоли / 2 бинки готовой
\item 1 морковка 
\item 1 луковица
\item 3 сладких перца
\item 3 помидора
\item 100 г грецких орехов 
\item 2 стл толёнки 
}{
\item 2 лаврового листа
\item 0.5 чл сухово базилика / три веточки свежей
\item 0.5 чл хмели-сунели
\item 2 чл кориандр
\item 3 щеп чили 
\item 3 зуб чеснока 
\item 0.5 пучка кинзы
\item 1 чл соли
}{
Фасоль сварить отдельно, либо ошпарить баночную. В казане обжарить морковь, лук и перец, затем закинуть всё остальное и потушить вместе. Зелень, чеснок и соль за 3 минуты до окончания.
}{}{lobio}

%%% ================== recipe ================== %%%
\recipe{Гречка «Намасте»}{25}{1.7 кг (4 порции)}
{
\item 2 ст гречки
\item 200 г моркови
\item 200 г болгарского сладкого перца
\item 3,5 ст кипятка
\item горсть укропа
}{
\item 1 чл хмели-сунели
\item 1/2 чл кориандра
\item 1 щеп асафетиды / 1 долька чеснока
\item 1,5 чл соли
\item 2 чл сахара
}{
Промыть гречку, потереть морковь, нашинковать перец. Специи обжарить в масле несколько секунд, обжарить морковь и перец, добавить гречку, сахар-соль и залить кипятком. Варить под крышкой 10 минут на сильном огне, 6 минут на тихом. Укроп в конце.
}{
\begin{advice}
\item Если лень~--- можно обжарить только специи.
\item Для меня, гречка ~--- это лидер среди круп, а по факту — семян. В принципе, из всех круп можно было бы оставить лишь её одну. Она самая сытная, доступная и почти не засоряет ЖКТ. 
\item За рецепт благодарность и хвала мастеру ведической науки. 
\end{advice}}{}



%%% ================== recipe ================== %%%
\recipe{Живая гречка}{}{1 кг (2-3 порции)}
{
\item гречка зелёная, 1 ст
\item помидоры черри или вкусные сезонные 400
\item петрушка, кинза или руккола 50
}{
\item лимонный сок 2 стл
\item соль 1 чл
}{
Гречку прорастить. Делается это так: промыть, залить холодной водой на 20-30 минут, слить, промыть, высыпать в пластиковый дуршлаг с поддоном (найдите на рынке такой, чтобы в отверстия не проваливались зёрна), распределить и накрыть чем угодно, оставив щель толщиной с палец (для аэрации). Через $\pm18$ часов гречка проросла.

В результате имеем 2 стакана. Смешиваем с остальным, получается вкуснейшая вещь! Содержит всё необходимое для человека.
}{
\begin{advice}
    \item Пропорции примерные.
\item Если растёт медленно, значит в помещении холодно.
    \item Если подгнивает~--- значит либо слишком жарко в помещении, либо щель не оставили и она задохнулась, либо передержали в воде.
\item Проращивайте заранее, остатки храните в холодильнике в контейнере.
    \item Ещё один образец суперфуда, который можно рассматривать как ежедневное блюдо. Поскольку сама по себе она практически безвкусная, то вкус надо смоделировать.
\end{advice}}{grech}



%%% ================== recipe ================== %%%
\recipe{Ароматный рис}{15}{6 порций}
{
\item рис жасмин или басмати непропаренный 2 ст
\item кипяток 4 ст
\item масло топленое 3 стл
\item кешью горсть
\item пучок петрушки
}{
\item кумин 2 чл
\item корица 1 чл
\item мускат 1/2 чл
\item соль 1,5 чл
\item сахар 1 чл
}{
Рис промыть. Обжарить кешью, кумин, всыпать рис, кипяток и специи. Варить на тихом огне под крышкой 15 минут без вмешательства! Зелень в конце.
}{}{}



%%% ================== recipe ================== %%%
\recipe{Вега-плов}{}{}
{
\item рис жасмин или басмати непропаренный 2 ст
\item кипяток 4 ст
\item масло топленое 3 стл
\item кешью горсть
\item горсть изюма 
\item горсть кураги
\item большая морковь
\item большая луковица
\item пучок петрушки 
}{
\item зира (кумин) 1 чл
\item барбарис 1 чл
\item листья сафлора 1/2 чл
\item сушёные помидоры или сушёный сладкий перец 1/2 чл
\item куркума 1/4 чл
\item целая головка чеснока
}{
Рис промыть. Обжарить кешью, кумин, всыпать рис, кипяток и специи. Варить на тихом огне под крышкой 15 минут без вмешательства! Зелень в конце
}{
\begin{advice}
\item Курагу я бы рекомендовал замочить заранее и добавить в конце (иначе с ней рис при варке превратится в кисель). 
\item А вот изюм можно добавить после обжарки моркови и лука перед закладкой риса. 
\item Вообще рис — это сплошной крахмал, поэтому не увлекайтесь.
\end{advice}
}{}



%%% ================== recipe ================== %%%
\recipe{Овсяночка!}{}{3 порции}
{
\item 1 ст овсяных хлопьев 2-минутных «экспресс»
\item 2 ст воды 
}{
\item 1 щепотка соли, 
\item 2 чл сахара с горкой, 
\item 1 щепотки ванили, 
\item 2 щепотки корицы
}{
Заливаете \textbf{холодной} водой, специи, до кипения, постоянно помешивая. 2 минуты варки и готово. Можно сливки добавить, сухофрукты и всё, на что фантазии хватит. 
}{
\begin{advice}
\item Куда же без овсянки. Знаю людей, которые заставили себя её полюбить, потому что спорт требует. Предлагаю обойтись без насилия и просто научиться её вкусно готовить.
\item Из доступных и полезных круп, гречка и овсянка, пожалуй, лидеры. Остальные интереса не представляют и рассказывать я о них не буду.
\item Только со свежими фруктами не мешайте, потому что не сочетаются. С сухофруктами тоже не очень, честно говоря. Смотри таблицу раздельного питания. 
    \item Если хотите~--- попробуй с орехами и брынзой, как несладкий вариант.
\item Ешьте овсянку, будьте сильными!
\end{advice}}{}



%%% ================== recipe ================== %%%
\recipe{Льняная сладкая каша за 30 с}{}{1 порция}
{
\item 3 стл белого льна
\item 1 стл мёда или 4 финика
\item 1 ст воды
\item 1 стл любых ягод или сухофруктов на вкус и цвет
}{
\item Мед не специя \faSmileO
}{
Всё перемалывается в мощном блендере. Если выдержать минут 15, то загустеет.
}{
\begin{advice}
\item Лён исключительно белый, потому что нежнее, вкуснее и не пахнет рыбой. 
\item Натуральный бюджетный источник ПНЖК, чисто функциональный продукт, который иногда надо включать в рацион. 
\item Если лён ну никак не лезет, выбирайте авокадо или грецкий орех.
\end{advice}}{len}




%%% ================== recipe ================== %%%
\recipe{Фасоль с орешками}{}{400г (1 порция)}
{
\item фасоль зелёная стручковая 400
\item кедровые орехи, горсть
\item топлёнка, 2 стл
}{
\item ---
}{
Обжарить.
}{
\begin{advice}
\item Самый короткий рецепт — для самых продвинутых!
\end{advice}}{beans}



%%% ================== recipe ================== %%%
\recipe{Крахмалистые тушёные овощи}{}{1}
{
\item 400 г тыквы
\item 400 г кабачка
\item 40 г моркови
\item 1 горсть кешью
\item петрушка или кинза
\item 2 стл топленого масла
}{
\item 1 чл соли
\item 1 чл кориандра
\item 0.5 чл кумина
\item 0.5 чл куркумы
\item 0.5 чл асафетиды / 1 долька чеснока
}{
Овощи режем грубо. В сковороде или казане жарим орешки и специи в масле, закидываем чищеные баклажаны, потом остальные овощи в указанном порядке. Если надо пропарить, закройте крышкой. Добавить рубленую зелень, чеснок, посолить. Брынза по тарелкам.
}{
\begin{advice}
\item Куркуму заменить на корицу — будет «потеплее».
\end{advice}}{}



%%% ================== recipe ================== %%%
\recipe{Пхали — грузинская закуска}{10}{0.3}
{
\item 1 стакан грецких орехов
\item 1 пучок кинзы
\item 1 пучок шпината
\item 2 столовые ложки лимонного сока
}{
\item 0.5 зуб чеснока
\item 0.5 чл соли
}{
Все смешать в мельничке или мясорубке до консистенции паштета. Супер-вкусно!
}{
\begin{advice}
\item Можно накатать кругляшей и подавать на листьях шпината.
\item Или положить в контейнер и намазывать на хлебцы.
\end{advice}}{balls}



