\chapter*{Напитки}
\label{sec:drinks}
\addcontentsline{toc}{section}{\nameref{sec:drinks}}


%%% ================== recipe ================== %%%
\recipe{Чай масала}{15}{1.3}
{
\item 3 ст кипятка
\item 2 ст нежирного молока
\item Свежий имбирь $1\times 3$ см (1 стл с горкой, 10г)
\item 3 стл (40г) сахарного песка
\item 3 стл с горкой (16г) черного чая <<Пуэр>>
\item Марлевые салфетки или мелкое сито
}{
\item 0.25 мускатного ореха
\item 6-8 (0.25 чл молотой) веточек гвоздики 
\item Корица молотая 1 чл (2г)
\item 20 семян (2 чл молотого) зеленого кардамона  (раздавить в ступе)  
\item 1 чёрный кардамон 
}{
В кастрюле залить кипятком специи, сахар и имбирь, варить 5 минут, всыпать чай и поварить ещё 5 минут. Влить молоко и довести до кипения. Процедить напиток через сито или марлю. Подавать горячим.
}{
\begin{advice}
\item Рецепт проверен временем и выверен по пропорциям. Для романтического вечера — самое то!
\item С осторожностью обходитесь с имбирём. Он усиливает не только острый запах кардамона, но и даёт острый вкус. Пробуйте напиток в процессе! Молоко почти не смягчает остроту. 
    \item Сушеный имбирь использовать не рекомендую.
    \item Для веганского варианта то же самое, только варить всё 10 минут без чая и молока (заменить просто водой), процедить и мёд в конце.
            \item Пуэр можно найти в любом чайном магазине или магазине специй. Сейчас проблем с этим нет. Берите, чтоб нравился аромат — обычно пахнет хорошим чернозёмом.
                \item Почему я отказался от чая? Потому что считаю его сильным закислителем и, по сути, таким же наркотиком, как и кофе.
\end{advice}
\newpage
~
}{tea}





%%% ================== recipe ================== %%%
\recipe{Яблочный напиток}{10}{0.75}
{
\item 1 яблоко
\item 500 кипятка
\item 1 стл сиропа шиповника («Холосас» из аптеки / 0.5 ст крепкого отвара шиповника + 1 чл сахара)

}{
\item 2 палочки корицы или 2 щепотки
\item 3 веточки гвоздики
\item 1 звездочка бадьяна
\item 1 щепотка черного перца
\item 1/2 чл ванильного экстракта или ванили
}{%
}{}{apple}




%%% ================== recipe ================== %%%
\recipe{Напиток из пажитника}{20}{1}
{
\item 1 л кипятка
\item мёд по вкусу
}{
\item 2 стл смян пажитника
}{
Семена варить 15 минут, процедить отвар и добавить мёд.
}{
\begin{advice}
\item Если выпить всю кастрюлю, 3 дня будете пахнуть пажитником \faSmileO. Благо аромат приятный \faSmileO.
\item Можно вообще на варить, а залить кипятком из расчёта 1 чл семян на стакан кипятка. Через 5 минут готов. Можно заваривать 2–3 раза. 
\end{advice}}{fenugreek}



%%% ================== recipe ================== %%%
\recipe[]{Напиток из шиповника}{100(10 работы)}{2}
{
\item 2 ст шиповника
\item 2 л воды
}{
\item[] 
}{
Шиповник варить в скороварке 1,5 ч.
}{
\begin{advice}
    \item Самый полезный напиток.
    \item Шиповник — классическое желчегонное наравне с брусникой.
\item Получается концентрированный отвар, который можно разбавлять и упиваться вместе с мёдом в течение дня и после 5 вечера, когда рабочий день поджелудочной уже закончен. 

\end{advice}}{}





%%% ================== recipe ================== %%%
\recipe{Королевский чай}{15}{1}
{
\item  1 л кипятка
}{
\item 1 пакетик (125 мг) шафрана 
\item 10 коробочек зелёного кардамона
\item 2 палочки цейлонской корицы
}{
Залить кипятком на 10 минут. Корицу вынуть.
}{
\begin{advice}
\item Рецепт, которого нет. Этот чаёк когда-то делали в Джаге на Кузнецком. Просто бомба! Попробуйте хотя бы раз в жизни. 
 \item Почему он так называется, вы поймёте, прочитав состав и прикинув стоимость \faSmileO.
\end{advice}}{royal}





%%% ================== recipe ================== %%%
\recipe{Напиток из базилика}{15}{2}
{
\item 2 л кипятка
\item 0.5 лимона 
\item Мёд по вкусу / 0.33 ст сахара
}{
\item 1 пучок базилика
}{
Базилик залить на 15 минут. Процедить, добавить лимон и мёд. Подавать горячим или холодным.
}{}{basil}





%%% ================== recipe ================== %%%
\recipe{Напиток из кураги}{15}{2}
{
\item 300 г мягкой кураги
\item 50 г мёда
\item 2 л кипятка
}{
\item 0.5 пучка ($\approx 20$ г) мяты  / 2-3 г сухой
}{
Курагу варить 10 минут, добавить мяту, настоять. Мёд в конце.
}{}{}




%%% ================== recipe ================== %%%
\recipe[]{Имбирник}{15}{2}
{
\item Свежий имбирь $6\times 2$ см (4 стл с горкой, 40г)
\item 100 г мёда
\item 2 стл лимонного сока
\item 2 л кипятка
}{
\item[] 
}{
Мелко потереть имбирь, варить 10 минут, процедить. Влить лимон и мёд, когда остынет до $60^{\circ}$С. Можно при варке добавить каркаде, получится ещё интересней.

}{
\begin{advice}
\item Добавьте в напиток чёрного перца. Он откроет доступ для имбиря к вашим клеткам.
\end{advice}}{ginger}


% %%% ================== recipe ================== %%%
% \recipe{}{}{}
% {
% \item 
% }{
% \item 
% }{
% }{}{}
% 
% 
% 
% 
