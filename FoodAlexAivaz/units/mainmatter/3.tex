\chapter{Рецепты}

\chapter*{Супы}
\label{sec:soup}
\addcontentsline{toc}{section}{\nameref{sec:soup}}

\newpage
\subsection*{\Large Чечевичный суп}
\label{sec:soupLent}
\addcontentsline{toc}{subsection}{\nameref{sec:soupLent}}
\faClockO\ 20 мин.

\textbf{\large Ингредиенты}
\begin{itemize}
\item 1/2 ст красной чечевицы
\item 3 ст кипятка
\item 2 горсти нарезанных помидоров
\end{itemize}

\textbf{Специи}
\begin{itemize}
\item 1/2 чл сахара (или 2 чл кокосового молока)
\item 1/2 чл зиры
\item 1 щепотка чили
\item 1 щепотка имбиря
\item 1 лавровый лист
\item 1 чл соли
\item 1 зуб чеснока
\item горсть кинзы
\end{itemize}

\textbf{\large Метод}

Чечевицу промыть, отварить со специями 12-15 минут. Помидоры и соль до кипения, выключить. Чеснок и зелень~--- в конце. % what? correct


\newpage
\section*{\Large Тыквенный суп}
\label{sec:soup2}
\addcontentsline{toc}{subsection}{\nameref{sec:soup2}}

\faClockO\ 5 мин.

\textbf{\large Ингридиенты}
\begin{itemize}
\item 2 ст. резаной тыквы
\item 2 ст. воды
\item 2 чл топлёнки
\item Горсть, тыквенные семечки 
\item 50 мл сливки (по желанию)
\item Зелень
\end{itemize}

\textbf{Специи}
\begin{itemize}
\item 1 чл тёртого имбиря
\item 1/4 чл зёрен горчицы
\item 1/4 чл куркумы
\item 2 щеп. черного перца
\item 2 щеп. муската
\item 1 чл соли
\end{itemize}


\textbf{\large Метод}

Тыкву в блендер. Обжарить в масле горчицу и имбирь 20 сек, добавить тыкву и специи, залить кипятком и поварить минуту. Выключить, разбить блендером, добавить сливки, семечки и рубленую зелень. 









\newpage
\section*{\Large Бобовый суп~--- самый сытный из супов}
\label{sec:soup2}
\addcontentsline{toc}{subsection}{\nameref{sec:soup2}}

\faClockO\ ? мин. \hfill \faSpoon\ 3 литра (7–10 порций)


\textbf{\large Ингридиенты}
\begin{itemize}
\item 4 банки консервированных красных бобов или 1,5 ст сухих
\item 5 ст кипятка
\item 1 стл растительного масла
\item 1 средняя луковица (по желанию)
\item 8 помидоров (ломтиками)
\item гарнир: петрушка, брынза
\end{itemize}

\textbf{Специи}
\begin{itemize}
\item 1 стл сушёного сладкого перца или помидоров
\item 1/4 чл сушёного острого красного перца
\item 1/2 чл розового молотого перца
\item 1 чл сухого орегано
\item 1 чл кумина (зиры)
\item 1 стл соли без горки
\item 3 зубчика чеснока (мелко)
\end{itemize}
\begin{formal}
\begin{itemize}
\item Специи лучше измолоть.
\end{itemize}
\end{formal}

\textbf{\large Метод}

Фасоль промыть, залить кипятком, довести до кипения, слить, промыть. Влить 5 ст. свежего кипятка, специи и на тихий огонь под крышку на 20 мин. В процессе пожарить лук, добавить. Подготовить помидоры, чеснок, соль, и добавить за 2 минуты. Подавать со свежей петрушкой и брынзой.

Если делаете из сухой фасоли, то замачивается на 6–8 часов, меняется вода, до кипения, вновь свежая вода и варится 1 час, либо экспресс методом по схеме 3+1: залить холодной водой и поварить 5 минут, оставить на 3 часа, слить и залить свежей водой, довести до кипения, вновь слить, залить свежей водой и варить 1 час.

 



\newpage
\section*{\Large }
\label{sec:soup2}
\addcontentsline{toc}{subsection}{\nameref{sec:soup2}}

\faClockO\ ? мин.

\textbf{\large Ингридиенты}
\begin{itemize}
\item 
\end{itemize}

\textbf{Специи}
\begin{itemize}
\item 
\end{itemize}


\textbf{\large Метод}

