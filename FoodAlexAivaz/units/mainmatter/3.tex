
\chapter*{Супы}
\label{sec:soup}
\addcontentsline{toc}{section}{\nameref{sec:soup}}


%%% ================== recipe ================== %%%

\recipe{Чечевичный суп}{20}{}
{
\item 1/2 ст красной чечевицы
\item 3 ст кипятка
\item 2 горсти нарезанных помидоров
}{
\item 1/2 чл сахара (или 2 чл кокосового молока)
\item 1/2 чл зиры
\item 1 щепотка чили
\item 1 щепотка имбиря
\item 1 лавровый лист
\item 1 чл соли
\item 1 зуб чеснока
\item горсть кинзы
}{
Чечевицу промыть, отварить со специями 12-15 минут. Добавить помидоры и соль и довести до кипения, после выключить. Чеснок и зелень~--- в конце. % what? correct
}{
\begin{advice}
\item Для тайско-индийского варианта добавить 2 стл кокосовой стружки и ещё 1/2 чл сахара.
\end{advice}}{soupLentils}




%%% ================== recipe ================== %%%
\recipe{Тыквенный суп}{5}{}
{
\item 2 ст. резаной тыквы
\item 2 ст. воды
\item 2 чл топлёнки
\item Горсть тыквенных семечек 
\item 50 мл сливки (по желанию)
\item Зелень
}{
\item 1 чл тёртого имбиря
\item 1/4 чл зёрен горчицы
\item 1/4 чл куркумы
\item 2 щеп. черного перца
\item 2 щеп. муската
\item 1 чл соли
}
{
Тыкву в блендер. Обжарить в масле горчицу и имбирь 20 сек, добавить тыкву и специи, залить кипятком и поварить минуту. Выключить, разбить блендером, добавить сливки, семечки и рубленую зелень. 
}{}{soup2}

%%% ================== recipe ================== %%%
\recipe{Бобовый суп }{}{3 кг}
{
\item 4 банки консервированных красных бобов или 1,5 ст сухих
\item 5 ст кипятка
\item 1 стл растительного масла
\item 1 средняя луковица (по желанию)
\item 8 помидоров (ломтиками)
\item гарнир: петрушка, брынза
}{
\item 1 стл сушёного сладкого перца или помидоров
\item 1/4 чл сушёного острого красного перца
\item 1/2 чл розового молотого перца
\item 1 чл сухого орегано
\item 1 чл кумина (зиры)
\item 1 стл соли без горки
\item 3 зубчика чеснока (мелко)
}{
Фасоль промыть, залить кипятком, довести до кипения, слить, промыть. Влить 5 ст. свежего кипятка, специи и на тихий огонь под крышку на 20 мин. В процессе пожарить лук, добавить. Подготовить помидоры, чеснок, соль, и добавить за 2 минуты. Подавать со свежей петрушкой и брынзой.

Если делаете из сухой фасоли, то замачивается на 6–8 часов, меняется вода, до кипения, вновь свежая вода и варится 1 час, либо экспресс методом по схеме 3+1: залить холодной водой и поварить 5 минут, оставить на 3 часа, слить и залить свежей водой, довести до кипения, вновь слить, залить свежей водой и варить 1 час.
}{
\begin{advice}
\item Специи лучше измолоть.
    \item Самый сытный из супов.
\end{advice}
}{}



%%% ================== recipe ================== %%%
\recipe{Суп-пюре из брокколи}{}{0.5 кг}
{
\item 1 стл топлёного масла
\item горсть кешью
\item 400г брокколи
\item 2 стл сливок
}{
\item 1 палочка корицы
\item 1/2 чл горчица
\item 1/2 чл кумин 
\item 1 долька чеснока
\item 1 щепотка кардамона
\item 1 щепотка имбиря 
\item 1 щепотка черного перца
}{
Обжарить в масле кешью и горчицу, затем кумин, туда же рубленую брокколи + кипяток, чтобы покрыл капусту, оставить на огне. Когда капуста сварится~--- пюрировать блендером и посолить. Добавить остальное, взбить и довести до кипения.
}{}{}


 
%%% ================== recipe ================== %%%
\recipe{Супчик из маша}{}{}
{
\item 100 г маша
\item 1 горсть моркови
\item 3 ст кипятка
\item 100 г помидоров
}{
\item 1/2 чл зиры
\item 1 лаврушка
\item 1 чл соли
\item 1 зуб чеснока
\item 1 горсть петрушки
}{
Маш промыть, морковь порезать кубиками, специи, варить 40 минут. Помидоры и соль до кипения, выключить. Чеснок и зелень в конце.
}{}{}




%%% ================== recipe ================== %%%
\recipe{Томатный суп-пюре}{}{1.5 кг}
{
\item мука 1 стл
\item топлёнка, 1 стл
\item помидоры, 8 шт
\item кипяток 1 ст
\item зелень
}{
\item соль, 1 чл
\item перец, 2 щеп
\item прованские травы, 1 чл или орегано, 0.5 чл
\item чеснок, 1 зуб
}{
В кастрюле муку обжарить в масле до золота, помидоры дольками, специи, воду, довести до кипения, разбить блендером, поварить 1 минуту. Подавать с зеленью, брынзой и чесночными ржаными хлебцами.
}{
\begin{advice}
\item Если вам надоело есть разведённую томатную пасту, что подают в рэсторанах~--- приготовьте настоящий суп-пюре самостоятельно.
\end{advice}}{}



