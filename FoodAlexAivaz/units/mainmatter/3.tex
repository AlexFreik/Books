
\chapter*{Супы}
\label{sec:soup}
\addcontentsline{toc}{section}{\nameref{sec:soup}}


%%% ================== recipe ================== %%%

\recipe{Чечевичный суп}{20}{1}
{
\item 0.5 ст красной чечевицы
\item 3 ст кипятка
\item 2 горсти нарезанных помидоров
}{
\item 0.5 чл сахара / 2 чл кокосового молока
\item 0.5 чл зиры
\item 1 щепотка чили
\item 1 щепотка имбиря
\item 1 лавровый лист
\item 1 чл соли
\item 1 зуб чеснока
\item 1 горсть кинзы
}{
Чечевицу промыть, отварить со специями 12-15 минут. Добавить помидоры и соль и довести до кипения, после выключить. Чеснок и зелень~--- в конце. % what? correct
}{
\begin{advice}
\item Для тайско-индийского варианта добавить 2 стл кокосовой стружки и ещё 1/2 чл сахара.
\end{advice}}{soupLentils}




%%% ================== recipe ================== %%%
\recipe{Тыквенный суп}{10}{0.8}
{
\item 2 ст. резаной тыквы
\item 2 ст. воды
\item 2 чл топлёнки
\item Горсть тыквенных семечек 
\item 50 мл сливки (по желанию)
\item Зелень
}{
\item 1 чл тёртого имбиря
\item 0.25 чл зёрен горчицы
\item 0.25 чл куркумы
\item 2 щеп. черного перца
\item 2 щеп. муската
\item 1 чл соли
}
{
Тыкву в блендер. Обжарить в масле горчицу и имбирь 20 сек, добавить тыкву и специи, залить кипятком и поварить минуту. Выключить, разбить блендером, добавить сливки, семечки и рубленую зелень. 
}{}{soup2}

%%% ================== recipe ================== %%%
\recipe{Бобовый суп}{40}{3}
{
\item 4 банки консервированных красных бобов или 1,5 ст сухих
\item 5 ст кипятка
\item 1 стл растительного масла
\item 1 средняя луковица (по желанию)
\item 8 помидоров (ломтиками)
\item гарнир: петрушка, брынза
}{
\item 1 стл сушёного сладкого перца или помидоров
\item 0.25 чл сушёного острого красного перца
\item 0.5 чл розового молотого перца
\item 1 чл сухого орегано
\item 1 чл кумина (зиры)
\item 1 стл соли без горки
\item 3 зубчика чеснока (мелко)
}{
Фасоль промыть, залить кипятком, довести до кипения, слить, промыть. Влить 5 ст. свежего кипятка, специи и на тихий огонь под крышку на 20 мин. В процессе пожарить лук, добавить. Подготовить помидоры, чеснок, соль, и добавить за 2 минуты. Подавать со свежей петрушкой и брынзой.

Если делаете из сухой фасоли, то замачивается на 6–8 часов, меняется вода, до кипения, вновь свежая вода и варится 1 час, либо экспресс методом по схеме 3+1: залить холодной водой и поварить 5 минут, оставить на 3 часа, слить и залить свежей водой, довести до кипения, вновь слить, залить свежей водой и варить 1 час.
}{
\begin{advice}
\item Специи лучше измолоть.
    \item Самый сытный из супов.
\end{advice}
}{beansSoup}



%%% ================== recipe ================== %%%
\recipe{Суп-пюре из брокколи}{}{0.5}
{
\item 1 стл топлёного масла
\item 1 горсть кешью
\item 400 г брокколи
\item 2 стл сливок
}{
\item 1 палочка корицы
\item 0.5 чл горчицы
\item 0.5 чл кумина
\item 1 долька чеснока
\item 1 щеп кардамона
\item 1 щеп имбиря 
\item 1 щеп черного перца
}{
Обжарить в масле кешью и горчицу, затем кумин, туда же рубленую брокколи и кипяток, чтобы покрыл капусту, оставить на огне. Когда капуста сварится~--- пюрировать блендером и посолить. Добавить остальное, взбить и довести до кипения.
}{}{broccoli}



%%% ================== recipe ================== %%%
\recipe{Супчик из маша}{45}{0.8}
{
\item 100 г маша
\item 1 горсть моркови
\item 3 ст кипятка
\item 100 г помидоров
}{
\item 1/2 чл зиры
\item 1 лаврушка
\item 1 чл соли
\item 1 зуб чеснока
\item 1 горсть петрушки
}{
Маш промыть, морковь порезать кубиками, специи, варить 40 минут. Помидоры и соль до кипения, выключить. Чеснок и зелень в конце.
}{}{mash}




%%% ================== recipe ================== %%%
\recipe{Томатный суп-пюре}{15}{1.5}
{
\item 1 стл муки 
\item 1 стл топленки
\item 8 шт помидоров
\item 1 ст кипятка
\item Зелень 
}{
\item 1 чл соли
\item 2 щеп перца
\item 1 чл прованских трав / 0.5 чл орегано
\item 1 зуб чеснока
}{
В кастрюле муку обжарить в масле до золота. Помидоры порезать дольками, и вместе со специями добавить в кипящую воду и довести до кипения, разбить блендером, поварить 1 минуту. Подавать с зеленью, брынзой и чесночными ржаными хлебцами.
}{
\begin{advice}
\item Если вам надоело есть разведённую томатную пасту, что подают в рэсторанах~--- приготовьте настоящий суп-пюре самостоятельно.
\end{advice}}{tomatoeSoup}



