\foodTitle{Десерты}{}


%%% ================== recipe ================== %%%
\recipe{Овсяный пирог}{35}{1}
{
\item 1 ст сахара
\item 1 ст кефира или сметаны
\item 90 г сливочного масла
\item 1.5 ст ц/з муки
\item 1.25 ст овсяных экспресс хлопьев % ???
\item горсть изюма
}{
\item 1 чл соды
\item 0.75 чл соли
\item 0.5 чл корицы
\item 1 чл ванильного сахара
}{
Смешать в указанном порядке. Вылить на противень, запечь $180^{\circ}$ 20 мин.
}{
\begin{advice}
\item Случай, когда гости приходят неожиданно.
\end{advice}}{}



%%% ================== recipe ================== %%%
\recipe{Полезное печенье}{5}{0.15}
{\label{cookies}
\item 100 г кунжута 
\item 50 г мёда
}{
\item 0.5 чл ванильного сахара
}{
Кунжут измельчить в кофемолке и все смешать.
}{
\begin{advice}
    \item  Жирненько, сладенько, пахнет выпечкой. По вкусу, как песочное тесто.
\item Если у вас аллергия на кунжут~--- покупайте чёрный нешлифованный~— он не отбелен известью. Если и на чёрный реакция, я вас поздравляю,~— вам надо очищаться, или же оставаться в комфортном болоте\ldots

\end{advice}}{}



%%% ================== recipe ================== %%%
\recipe[]{Живая халва за 5 минут}{5}{---}
{


\item подсолнух + мёд

    (можно с маком)
\item кунжут + мёд 

    (можно добавить ваниль)
\item абрикосовое ядро + мёд 

    (типичный марципан)

}{%
\item[]
}{
Принцип: семена + мёд в пропорции 2:1. Семена измельчаются в кофемолке. Варианты в ингредиентах.
Смешать в контейнере и там же хранить.
}{}{halva}



%%% ================== recipe ================== %%%
\recipe[]{Рафаэлло}{20}{0.2}
{
\item 100 г фиников

\item 50 г черногоизюма
\item 2 стл (20 г) кокосовой стружки 
\item Кокосовая стружка для обсыпки
}{
\item[]
    }{
Измельчить в мясорубке, смешать, накатать, обвалять в кокосовой стружке.  
}{
\begin{advice}
\item Можно ничего не катать и есть ложкой.
    \item Хранится в холоде.
        \item Брал финики Каспиан, изюм Изабелла. Стружку пробовать!

\end{advice}
}{raf}





%%% ================== recipe ================== %%%
\recipe{Апельсиновый шар {\normalsize (привет «Джаганнат»)}}{20}{0.15}
{
\item 100 г фиников
\item 40 мл зел. молотой гречки 
\item 2 чл какао
}{
\item 1 щеп свежей цедры апельсина
\item спичечная головка молотого кардамона
\item 1 щеп корицы
}{
Гречку в кофемолке, финики через мясорубку, всё смешать, накатать, обвалять в той же гречке или какао.
 
}{
\begin{advice}
\item Умножайте в несколько раз.
\item Пропорции не менять!
\end{advice}}{orangeBalls}




%%% ================== recipe ================== %%%
\recipe{Лимонные шары}{20}{0.13}
{
\item 100 г кураги
\item 20 г кокосовой стружки
\item 2 чл лимонного сока
\item Сверху сушёная лимонная цедра молотая
}{
\item 2 чл свежей мяты / 1 капля масла мяты
}{
Курагу через мясорубку, всё смешать, накатать, обвалять в цедре или стружке. 
}{
\begin{advice}
\item Умножайте в несколько раз.
\item Соблюдайте пропорции.
\end{advice}}{lemonBalls}







%%% ================== recipe ================== %%%
\recipe{Шоколад}{20}{0.1}
{
\item 10 г какао масло х/о
\item 100 г фиников
\item 5 г какао 
\item Молотая гречка для обсыпки
}{
\item  0.3 чл ванильной эссенции (концентрат)
}{
Гречку в кофемолке, финики через мясорубку, всё смешать, накатать, обвалять в гречке или какао. 
}{
\begin{advice}
\item Умножайте в несколько раз.
\item Соблюдайте пропорции.
\end{advice}}{choco}







% %%% ================== recipe ================== %%%
% \recipe{}{}{}
% {
% \item 
% }{
% \item 
% }{
% }{}{}
% 
% 
% 









