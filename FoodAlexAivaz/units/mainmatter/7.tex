\chapter*{Десерты}
\label{sec:sweet}
\addcontentsline{toc}{section}{\nameref{sec:sweet}}



%%% ================== recipe ================== %%%
\recipe{Овсяный пирог}{30}{1 кг}
{
\item 1 ст сахара
\item 1 ст кефира или сметаны
\item 90 г сливочного масла
\item 1,5 ст ц/з муки
\item 1,1/4 ст овсяных экспресс хлопьев % ???
\item горсть изюма
}{
\item 1 чл соды
\item 3/4 чл соли
\item 1/2 чл корицы
\item 1 чл ванильного сахара
}{
Смешать в указанном порядке. Вылить на противень, запечь $180^{\circ}$ 20 мин, порезать на куски.
}{
\begin{advice}
\item Случай, когда гости приходят неожиданно.
\end{advice}}{}



%%% ================== recipe ================== %%%
\recipe{Полезное печенье}{}{0.15 кг}
{\label{cookies}
\item кунжут 100 
\item мёд 50
}{
\item 1/2 чл ванильного сахара
}{
Кунжут измельчить в кофемолке и все смешать.
}{
\begin{advice}
    \item  Жирненько, сладенько, пахнет выпечкой. По вкусу, как песочное тесто.
\item Если у вас аллергия на кунжут~--- покупайте чёрный нешлифованный~— он не отбелен известью. Если и на чёрный реакция, я вас поздравляю,~— вам надо очищаться, или же оставаться в комфортном болоте\ldots

\end{advice}}{}



%%% ================== recipe ================== %%%
\recipe{Живая халва за 5 минут}{5}{}
{


\item подсолнух + мёд

    (можно с маком)
\item кунжут + мёд 

    (можно добавить ваниль)
\item абрикосовое ядро + мёд 

    (типичный марципан)

}{
\item ---
}{
Принцип: семена + мёд в пропорции 2:1. Семена измельчаются в кофемолке. Варианты в ингредиентах.
Смешать в контейнере и там же хранить.
}{}{halva}



%%% ================== recipe ================== %%%
\recipe{Рафаэлло}{}{0.2 кг}
{
\item финики 100
\item изюм чёрный 50
\item кокосовая стружка 2 стл (20г)
\item кокосовая стружка для обсыпки
}{
\item ---
}{
Измельчить в мясорубке, смешать, накатать, обвалять в кокосовой стружке.  
}{
\begin{advice}
\item Можно ничего не катать и есть ложкой.
    \item Хранится в холоде.
        \item Брал финики Каспиан, изюм Изабелла. Стружку пробовать!

\end{advice}
}{raf}





%%% ================== recipe ================== %%%
\recipe{Апельсиновый шар (привет «Джаганнат»)}{}{0.15 кг}
{
\item финики 100 г
\item зел гречка молотая, 40 мл
\item какао 2 чл
}{
\item цедра апельсина свежая, щепотка
\item кардамон молотый, со спичечную головку
\item корица, щепотка
}{
Гречку в кофемолке, финики через мясорубку, всё смешать, накатать, обвалять в той же гречке или какао.
 
}{
\begin{advice}
\item Умножайте в несколько раз.
\item Пропорции не менять!
\end{advice}}{orangeBalls}







%%% ================== recipe ================== %%%
\recipe{Лимонные шары}{}{}
{
\item курага 100
\item кокосовая стружка 20
\item лимонный сок 2 чл
\item сверху сушёная лимонная цедра молотая
}{
\item мята 2 щепотки (или 2 чл свежей или капля масла мяты)
}{
Курагу через мясорубку, всё смешать, накатать, обвалять в цедре или стружке. 
}{
\begin{advice}
\item Умножайте в несколько раз.
\item Пропорции не менять!
\end{advice}}{lemonBalls}







%%% ================== recipe ================== %%%
\recipe{}{}{}
{
\item 
}{
\item 
}{
}{}{}







%%% ================== recipe ================== %%%
\recipe{}{}{}
{
\item 
}{
\item 
}{
}{}{}







%%% ================== recipe ================== %%%
\recipe{}{}{}
{
\item 
}{
\item 
}{
}{}{}










