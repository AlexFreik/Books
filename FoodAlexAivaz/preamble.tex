%%% PACKAGES 
%%%------------------------------------------------------------------------------

\usepackage[utf8]{inputenc} % If utf8 encoding
% \usepackage[lantin1]{inputenc} % If not utf8 encoding, then this is probably the way to go
\usepackage[T1]{fontenc}    %
\usepackage[english,russian]{babel} % English please
\usepackage[final]{microtype} % Less badboxes

% \usepackage{kpfonts} %Font

\usepackage{amsmath,amssymb,mathtools} % Math

\usepackage[usenames,dvipsnames,svgnames,table,rgb]{xcolor}
\usepackage{tikz} % Figures. Following colors are predefined: red, green, blue, cyan, magenta, yellow, black, gray, darkgray, lightgray, brown, lime, olive, orange, pink, purple, teal, violet and white.

% Defining a new coordinate system for the page:
%
% --------------------------
% |(-1,1)    (0,1)    (1,1)|
% |                        |
% |(-1,0)    (0,0)    (1,0)|
% |                        |
% |(-1,-1)   (0,-1)  (1,-1)|
% --------------------------
\makeatletter
\def\parsecomma#1,#2\endparsecomma{\def\page@x{#1}\def\page@y{#2}}
\tikzdeclarecoordinatesystem{page}{
    \parsecomma#1\endparsecomma
    \pgfpointanchor{current page}{north east}
    % Save the upper right corner
    \pgf@xc=\pgf@x%
    \pgf@yc=\pgf@y%
    % save the lower left corner
    \pgfpointanchor{current page}{south west}
    \pgf@xb=\pgf@x%
    \pgf@yb=\pgf@y%
    % Transform to the correct placement
    \pgfmathparse{(\pgf@xc-\pgf@xb)/2.*\page@x+(\pgf@xc+\pgf@xb)/2.}
    \expandafter\pgf@x\expandafter=\pgfmathresult pt
    \pgfmathparse{(\pgf@yc-\pgf@yb)/2.*\page@y+(\pgf@yc+\pgf@yb)/2.}
    \expandafter\pgf@y\expandafter=\pgfmathresult pt
}
\makeatother

\usepackage{graphicx}  % Include figures
\usepackage{wrapfig}
\graphicspath{{img/}{../img/}}


%%% PAGE LAYOUT 
%%%------------------------------------------------------------------------------

\setlrmarginsandblock{0.15\paperwidth}{*}{1} % Left and right margin
\setulmarginsandblock{0.1\paperwidth}{*}{1}  % Upper and lower margin
\checkandfixthelayout

%%% SECTIONAL DIVISIONS
%%%------------------------------------------------------------------------------

\maxsecnumdepth{subsection} % Subsections (and higher) are numbered
\setsecnumdepth{subsection}

\makeatletter %
\makechapterstyle{standard}{
  \setlength{\beforechapskip}{0\baselineskip}
  \setlength{\midchapskip}{1\baselineskip}
  \setlength{\afterchapskip}{8\baselineskip}
  \renewcommand{\chapterheadstart}{\vspace*{\beforechapskip}}
  \renewcommand{\chapnamefont}{\centering\normalfont\Large}
  \renewcommand{\printchaptername}{\chapnamefont \@chapapp}
  \renewcommand{\chapternamenum}{\space}
  \renewcommand{\chapnumfont}{\normalfont\Large}
  \renewcommand{\printchapternum}{\chapnumfont \thechapter}
  \renewcommand{\afterchapternum}{\par\nobreak\vskip \midchapskip}
  \renewcommand{\printchapternonum}{\vspace*{\midchapskip}\vspace*{5mm}}
  \renewcommand{\chaptitlefont}{\centering\bfseries\LARGE}
  \renewcommand{\printchaptertitle}[1]{\chaptitlefont ##1}
  \renewcommand{\afterchaptertitle}{\par\nobreak\vskip \afterchapskip}
}
\makeatother

\chapterstyle{standard}

\setsecheadstyle{\normalfont\large\bfseries}
\setsubsecheadstyle{\normalfont\normalsize\bfseries}
\setparaheadstyle{\normalfont\normalsize\bfseries}
\setparaindent{0pt}\setafterparaskip{0pt}

%%% FLOATS AND CAPTIONS
%%%------------------------------------------------------------------------------

\makeatletter                  % You do not need to write [htpb] all the time
\renewcommand\fps@figure{htbp} %
\renewcommand\fps@table{htbp}  %
\makeatother                   %

\captiondelim{\space } % A space between caption name and text
\captionnamefont{\small\bfseries} % Font of the caption name
\captiontitlefont{\small\normalfont} % Font of the caption text

% \changecaptionwidth          % Change the width of the caption
% \captionwidth{1\textwidth} %

%%% ABSTRACT
%%%------------------------------------------------------------------------------

\renewcommand{\abstractnamefont}{\normalfont\small\bfseries} % Font of abstract title
\setlength{\absleftindent}{0.1\textwidth} % Width of abstract
\setlength{\absrightindent}{\absleftindent}

%%% HEADER AND FOOTER 
%%%------------------------------------------------------------------------------
\usepackage{nameref}
\makeatletter
\newcommand*{\currentname}{\@currentlabelname}
\makeatother

\makepagestyle{standard} % Make standard pagestyle

\makeatletter                 % Define standard pagestyle
\makeevenfoot{standard}{}{}{} %
\makeoddfoot{standard}{}{}{}  %
\makeevenhead{standard}{\bfseries\thepage\normalfont\qquad\small\leftmark}{}{}
\makeoddhead{standard}{}{}{\small\rightmark\qquad\bfseries\thepage}
% \makeheadrule{standard}{\textwidth}{\normalrulethickness}
\makeatother                  %

\makeatletter
\makepsmarks{standard}{
\createmark{chapter}{both}{shownumber}{\@chapapp\ }{ \quad }
\createmark{section}{right}{shownumber}{}{ \quad }
\createplainmark{toc}{both}{\contentsname}
\createplainmark{lof}{both}{\listfigurename}
\createplainmark{lot}{both}{\listtablename}
% \createplainmark{bib}{both}{\bibname}
\createplainmark{index}{both}{\indexname}
\createplainmark{glossary}{both}{\glossaryname}
}
\makeatother                               %

\makepagestyle{chap} % Make new chapter pagestyle

\makeatletter
\makeevenfoot{chap}{}{}{} % Define new chapter pagestyle
\makeoddfoot{chap}{}{}{}  %
\makeevenhead{chap}{\small\bfseries\thepage}{}{\hyperref[TOC]{Оглавление}}   %
\makeoddhead{chap}{\hyperref[TOC]{Оглавление}}{}{\small\bfseries\thepage}    %
\makeheadrule{chap}{\textwidth}{\normalrulethickness}
\makeatother

\nouppercaseheads
\pagestyle{standard}               % Choosing pagestyle and chapter pagestyle
\aliaspagestyle{chapter}{chap} %

%%% NEW COMMANDS
%%%------------------------------------------------------------------------------

\newcommand{\p}{\partial} %Partial
% Or what ever you want

%%% TABLE OF CONTENTS
%%%------------------------------------------------------------------------------

\maxtocdepth{subsection} % Only parts, chapters and sections in the table of contents
\settocdepth{subsection}

\AtEndDocument{\addtocontents{toc}{\par}} % Add a \par to the end of the TOC

%%% INTERNAL HYPERLINKS
%%%-------------------------------------------------------------------------------

\usepackage{hyperref}   % Internal hyperlinks
\hypersetup{
pdfborder={0 0 0},      % No borders around internal hyperlinks
pdfauthor={I am the Author} % author
}
% ----------- hyper ref -----------
\usepackage{hyperref}


\newcommand{\linkcolor}{blue}
\newcommand{\citecolor}{blue}
\newcommand{\filecolor}{magenta}
\newcommand{\urlcolor}{NavyBlue}
\hypersetup{				% Гиперссылки
	unicode=true,           % русские буквы в раздела PDF\\
	pdfstartview=FitH,
	colorlinks=true,  % false: ссылки в рамках; true: цветные ссылки
	linkcolor=\linkcolor,         % внутренние ссылки
	citecolor=\citecolor,        % на библиографию
	filecolor=\filecolor,      % на файлы
	urlcolor=\urlcolor,      % на URL
}
\usepackage{memhfixc}   %


%%% title page
\usepackage{pdfpages}
\ifpdf
  \usepackage{pdfcolmk}
\fi
%% check if using xelatex rather than pdflatex
\ifxetex
  \usepackage{fontspec}
\fi
\usepackage{graphicx}
%% for dingbats
\usepackage{pifont}
\newlength{\drop}% for my convenience

%% Generic publisher’s logo
\newcommand*{\plogo}{\fbox{$\mathcal{PL}$}}





\newcommand*{\titleJT}{\begingroup% Jan Tschichold: typographer
\drop = 0.08\textheight
\vspace*{\drop}
\hspace*{0.3\textwidth}
{\Large Александр Айвазян}\\[2\drop]
 \hspace*{0.3\textwidth}\begin{center}
         \Huge\itshape
         Здоровое питание:

         советы
         \&
         простые рецепты
\end{center}
%\hspace*{0.3\textwidth}{\Huge\itshape \textit{Здоровое питание: советы }}\par
%{\raggedleft\Huge\itshape простые рецепты\par}
\vfill
\hspace*{0.3\textwidth}{\includegraphics[width=0.1\linewidth]{img/isha_logo}}\\[0.5\baselineskip]
\hspace*{0.3\textwidth}{\Large }
\vspace*{\drop}
\endgroup}


%%% for beautiful icons
\usepackage{fontawesome}

\newenvironment{BiLines}
{\begin{samepage}\begin{center}
\begin{tikzpicture}
\draw[orange,thick] (0,0) -- (\textwidth,0);
\draw[orange,thick] (0,0.1) -- (\textwidth,0.1);
\end{tikzpicture}
\end{center}}
{\begin{center}
\begin{tikzpicture}
\draw[orange,thick] (0,0) -- (\textwidth,0);
\draw[orange,thick] (0,0.1) -- (\textwidth,0.1);
\end{tikzpicture}
\end{center}\end{samepage}}

\newenvironment{DidYouKnow}
{\begin{BiLines}
\textbf{\Large \faPencil\ Вы знали?}\\}
{\end{BiLines}}

\newenvironment{KeepInMind}
{\begin{BiLines}
\textbf{\Large \faLightbulbO\ Не забывайте}\\}
{\end{BiLines}}


%%% bibliography
\usepackage[backref=true]{biblatex}
\addbibresource{units/backmatter/mybib.bib}
\usepackage{csquotes}  % recomended to add

%%% indenting
\setlength{\parindent}{0em}
\setlength{\parskip}{0.5em}


%%%%%%%% Nice quote for recipes advice
% for adjustwidth environment
% \usepackage[strict]{changepage}

% for formal definitions
\usepackage{framed}

% environment derived from framed.sty: see leftbar environment definition
\definecolor{formalshade}{rgb}{0.95,0.95,1}

\newenvironment{formal}{%
  \def\FrameCommand{%
    \hspace{1pt}%
    {\color{darkgray}\vrule width 2pt}%
    {\color{formalshade}\vrule width 4pt}%
    \colorbox{formalshade}%
  }%
  \MakeFramed{\advance\hsize-\width\FrameRestore}%
  \noindent\hspace{-4.55pt}% disable indenting first paragraph
  \begin{adjustwidth}{}{7pt}%
  \vspace{2pt}\vspace{2pt}%
}
{%
  \vspace{2pt}\end{adjustwidth}\endMakeFramed%
}


%%%%
\usepackage{multicol}
\usepackage{paracol}
\usepackage{ifthen} 
\newenvironment{advice}
{%
\begin{formal}
\begin{itemize}%
}{%
\end{itemize}
\end{formal}%
}
%%%%% to center unnumbered subsections to create fancy recipe
% \usepackage{titlesec}
% % from the titlesec package
% %\titleformat{ command }
% %             [ shape ]
% %             { format }{ label }{ sep }{ before-code }[ after-code ]
% % \section*
% \titleformat{name=\subsection,numberless}
% {\filcenter\Large\bfseries}{}{0pt}{}
% %{\vspace{-15pt} \hrulefill}{-13pt}    {\scshape\LARGE\bfseries}[\vspace{-25pt}]
% \titlespacing*{name=\subsection,numberless}{0pt}{0pt}{-5pt}


\newcommand{\recipe}[9][Специи]
{%
\newpage
\subsection*{\emph{\huge #2}}
\addcontentsline{toc}{subsection}{#2}
%%% try to use just text, instead of sec, but sec is OK
% \par
% \noindent\makebox[\textwidth][c]{%
%     \begin{minipage}{0.6\textwidth}
%         \centering
%         \textbf{\emph{\huge #1}}
%     \end{minipage}
% 
% 
% }
% {\vspace{-30pt}

\faClockO\ \ifthenelse{\equal{#3}{}}{?}{#3} мин \hfill \faSpoon\ \ifthenelse{\equal{#4}{}}{?}{#4} кг
% }

\begin{tikzpicture}
\draw[gray,thick] (0,0) -- (\textwidth,0);
\draw[gray,thick] (0,0.1) -- (\textwidth,0.1);
\end{tikzpicture}

\begin{multicols}{2}
\textbf{\large Ингредиенты}
\begin{itemize}
#5
\end{itemize}
\vspace{100pt}

\columnbreak
\textbf{#1}
\begin{itemize}
#6
\end{itemize}
\end{multicols}
\nointerlineskip

\ifthenelse{\equal{#7}{}}{}{
\textbf{\large Метод}

#7
}
#8

\ifthenelse{\equal{#9}{}}{}{%
% \tikz[remember picture,overlay] \node[opacity=0.20,inner sep=0pt] at (page cs:0,-0.5){\includegraphics[height=0.5\paperwidth]{#8}};%
} 
}



